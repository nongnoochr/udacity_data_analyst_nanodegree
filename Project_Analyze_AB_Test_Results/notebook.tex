
% Default to the notebook output style

    


% Inherit from the specified cell style.




    
\documentclass[11pt]{article}

    
    
    \usepackage[T1]{fontenc}
    % Nicer default font (+ math font) than Computer Modern for most use cases
    \usepackage{mathpazo}

    % Basic figure setup, for now with no caption control since it's done
    % automatically by Pandoc (which extracts ![](path) syntax from Markdown).
    \usepackage{graphicx}
    % We will generate all images so they have a width \maxwidth. This means
    % that they will get their normal width if they fit onto the page, but
    % are scaled down if they would overflow the margins.
    \makeatletter
    \def\maxwidth{\ifdim\Gin@nat@width>\linewidth\linewidth
    \else\Gin@nat@width\fi}
    \makeatother
    \let\Oldincludegraphics\includegraphics
    % Set max figure width to be 80% of text width, for now hardcoded.
    \renewcommand{\includegraphics}[1]{\Oldincludegraphics[width=.8\maxwidth]{#1}}
    % Ensure that by default, figures have no caption (until we provide a
    % proper Figure object with a Caption API and a way to capture that
    % in the conversion process - todo).
    \usepackage{caption}
    \DeclareCaptionLabelFormat{nolabel}{}
    \captionsetup{labelformat=nolabel}

    \usepackage{adjustbox} % Used to constrain images to a maximum size 
    \usepackage{xcolor} % Allow colors to be defined
    \usepackage{enumerate} % Needed for markdown enumerations to work
    \usepackage{geometry} % Used to adjust the document margins
    \usepackage{amsmath} % Equations
    \usepackage{amssymb} % Equations
    \usepackage{textcomp} % defines textquotesingle
    % Hack from http://tex.stackexchange.com/a/47451/13684:
    \AtBeginDocument{%
        \def\PYZsq{\textquotesingle}% Upright quotes in Pygmentized code
    }
    \usepackage{upquote} % Upright quotes for verbatim code
    \usepackage{eurosym} % defines \euro
    \usepackage[mathletters]{ucs} % Extended unicode (utf-8) support
    \usepackage[utf8x]{inputenc} % Allow utf-8 characters in the tex document
    \usepackage{fancyvrb} % verbatim replacement that allows latex
    \usepackage{grffile} % extends the file name processing of package graphics 
                         % to support a larger range 
    % The hyperref package gives us a pdf with properly built
    % internal navigation ('pdf bookmarks' for the table of contents,
    % internal cross-reference links, web links for URLs, etc.)
    \usepackage{hyperref}
    \usepackage{longtable} % longtable support required by pandoc >1.10
    \usepackage{booktabs}  % table support for pandoc > 1.12.2
    \usepackage[inline]{enumitem} % IRkernel/repr support (it uses the enumerate* environment)
    \usepackage[normalem]{ulem} % ulem is needed to support strikethroughs (\sout)
                                % normalem makes italics be italics, not underlines
    

    
    
    % Colors for the hyperref package
    \definecolor{urlcolor}{rgb}{0,.145,.698}
    \definecolor{linkcolor}{rgb}{.71,0.21,0.01}
    \definecolor{citecolor}{rgb}{.12,.54,.11}

    % ANSI colors
    \definecolor{ansi-black}{HTML}{3E424D}
    \definecolor{ansi-black-intense}{HTML}{282C36}
    \definecolor{ansi-red}{HTML}{E75C58}
    \definecolor{ansi-red-intense}{HTML}{B22B31}
    \definecolor{ansi-green}{HTML}{00A250}
    \definecolor{ansi-green-intense}{HTML}{007427}
    \definecolor{ansi-yellow}{HTML}{DDB62B}
    \definecolor{ansi-yellow-intense}{HTML}{B27D12}
    \definecolor{ansi-blue}{HTML}{208FFB}
    \definecolor{ansi-blue-intense}{HTML}{0065CA}
    \definecolor{ansi-magenta}{HTML}{D160C4}
    \definecolor{ansi-magenta-intense}{HTML}{A03196}
    \definecolor{ansi-cyan}{HTML}{60C6C8}
    \definecolor{ansi-cyan-intense}{HTML}{258F8F}
    \definecolor{ansi-white}{HTML}{C5C1B4}
    \definecolor{ansi-white-intense}{HTML}{A1A6B2}

    % commands and environments needed by pandoc snippets
    % extracted from the output of `pandoc -s`
    \providecommand{\tightlist}{%
      \setlength{\itemsep}{0pt}\setlength{\parskip}{0pt}}
    \DefineVerbatimEnvironment{Highlighting}{Verbatim}{commandchars=\\\{\}}
    % Add ',fontsize=\small' for more characters per line
    \newenvironment{Shaded}{}{}
    \newcommand{\KeywordTok}[1]{\textcolor[rgb]{0.00,0.44,0.13}{\textbf{{#1}}}}
    \newcommand{\DataTypeTok}[1]{\textcolor[rgb]{0.56,0.13,0.00}{{#1}}}
    \newcommand{\DecValTok}[1]{\textcolor[rgb]{0.25,0.63,0.44}{{#1}}}
    \newcommand{\BaseNTok}[1]{\textcolor[rgb]{0.25,0.63,0.44}{{#1}}}
    \newcommand{\FloatTok}[1]{\textcolor[rgb]{0.25,0.63,0.44}{{#1}}}
    \newcommand{\CharTok}[1]{\textcolor[rgb]{0.25,0.44,0.63}{{#1}}}
    \newcommand{\StringTok}[1]{\textcolor[rgb]{0.25,0.44,0.63}{{#1}}}
    \newcommand{\CommentTok}[1]{\textcolor[rgb]{0.38,0.63,0.69}{\textit{{#1}}}}
    \newcommand{\OtherTok}[1]{\textcolor[rgb]{0.00,0.44,0.13}{{#1}}}
    \newcommand{\AlertTok}[1]{\textcolor[rgb]{1.00,0.00,0.00}{\textbf{{#1}}}}
    \newcommand{\FunctionTok}[1]{\textcolor[rgb]{0.02,0.16,0.49}{{#1}}}
    \newcommand{\RegionMarkerTok}[1]{{#1}}
    \newcommand{\ErrorTok}[1]{\textcolor[rgb]{1.00,0.00,0.00}{\textbf{{#1}}}}
    \newcommand{\NormalTok}[1]{{#1}}
    
    % Additional commands for more recent versions of Pandoc
    \newcommand{\ConstantTok}[1]{\textcolor[rgb]{0.53,0.00,0.00}{{#1}}}
    \newcommand{\SpecialCharTok}[1]{\textcolor[rgb]{0.25,0.44,0.63}{{#1}}}
    \newcommand{\VerbatimStringTok}[1]{\textcolor[rgb]{0.25,0.44,0.63}{{#1}}}
    \newcommand{\SpecialStringTok}[1]{\textcolor[rgb]{0.73,0.40,0.53}{{#1}}}
    \newcommand{\ImportTok}[1]{{#1}}
    \newcommand{\DocumentationTok}[1]{\textcolor[rgb]{0.73,0.13,0.13}{\textit{{#1}}}}
    \newcommand{\AnnotationTok}[1]{\textcolor[rgb]{0.38,0.63,0.69}{\textbf{\textit{{#1}}}}}
    \newcommand{\CommentVarTok}[1]{\textcolor[rgb]{0.38,0.63,0.69}{\textbf{\textit{{#1}}}}}
    \newcommand{\VariableTok}[1]{\textcolor[rgb]{0.10,0.09,0.49}{{#1}}}
    \newcommand{\ControlFlowTok}[1]{\textcolor[rgb]{0.00,0.44,0.13}{\textbf{{#1}}}}
    \newcommand{\OperatorTok}[1]{\textcolor[rgb]{0.40,0.40,0.40}{{#1}}}
    \newcommand{\BuiltInTok}[1]{{#1}}
    \newcommand{\ExtensionTok}[1]{{#1}}
    \newcommand{\PreprocessorTok}[1]{\textcolor[rgb]{0.74,0.48,0.00}{{#1}}}
    \newcommand{\AttributeTok}[1]{\textcolor[rgb]{0.49,0.56,0.16}{{#1}}}
    \newcommand{\InformationTok}[1]{\textcolor[rgb]{0.38,0.63,0.69}{\textbf{\textit{{#1}}}}}
    \newcommand{\WarningTok}[1]{\textcolor[rgb]{0.38,0.63,0.69}{\textbf{\textit{{#1}}}}}
    
    
    % Define a nice break command that doesn't care if a line doesn't already
    % exist.
    \def\br{\hspace*{\fill} \\* }
    % Math Jax compatability definitions
    \def\gt{>}
    \def\lt{<}
    % Document parameters
    \title{Analyze\_ab\_test\_results\_notebook}
    
    
    

    % Pygments definitions
    
\makeatletter
\def\PY@reset{\let\PY@it=\relax \let\PY@bf=\relax%
    \let\PY@ul=\relax \let\PY@tc=\relax%
    \let\PY@bc=\relax \let\PY@ff=\relax}
\def\PY@tok#1{\csname PY@tok@#1\endcsname}
\def\PY@toks#1+{\ifx\relax#1\empty\else%
    \PY@tok{#1}\expandafter\PY@toks\fi}
\def\PY@do#1{\PY@bc{\PY@tc{\PY@ul{%
    \PY@it{\PY@bf{\PY@ff{#1}}}}}}}
\def\PY#1#2{\PY@reset\PY@toks#1+\relax+\PY@do{#2}}

\expandafter\def\csname PY@tok@w\endcsname{\def\PY@tc##1{\textcolor[rgb]{0.73,0.73,0.73}{##1}}}
\expandafter\def\csname PY@tok@c\endcsname{\let\PY@it=\textit\def\PY@tc##1{\textcolor[rgb]{0.25,0.50,0.50}{##1}}}
\expandafter\def\csname PY@tok@cp\endcsname{\def\PY@tc##1{\textcolor[rgb]{0.74,0.48,0.00}{##1}}}
\expandafter\def\csname PY@tok@k\endcsname{\let\PY@bf=\textbf\def\PY@tc##1{\textcolor[rgb]{0.00,0.50,0.00}{##1}}}
\expandafter\def\csname PY@tok@kp\endcsname{\def\PY@tc##1{\textcolor[rgb]{0.00,0.50,0.00}{##1}}}
\expandafter\def\csname PY@tok@kt\endcsname{\def\PY@tc##1{\textcolor[rgb]{0.69,0.00,0.25}{##1}}}
\expandafter\def\csname PY@tok@o\endcsname{\def\PY@tc##1{\textcolor[rgb]{0.40,0.40,0.40}{##1}}}
\expandafter\def\csname PY@tok@ow\endcsname{\let\PY@bf=\textbf\def\PY@tc##1{\textcolor[rgb]{0.67,0.13,1.00}{##1}}}
\expandafter\def\csname PY@tok@nb\endcsname{\def\PY@tc##1{\textcolor[rgb]{0.00,0.50,0.00}{##1}}}
\expandafter\def\csname PY@tok@nf\endcsname{\def\PY@tc##1{\textcolor[rgb]{0.00,0.00,1.00}{##1}}}
\expandafter\def\csname PY@tok@nc\endcsname{\let\PY@bf=\textbf\def\PY@tc##1{\textcolor[rgb]{0.00,0.00,1.00}{##1}}}
\expandafter\def\csname PY@tok@nn\endcsname{\let\PY@bf=\textbf\def\PY@tc##1{\textcolor[rgb]{0.00,0.00,1.00}{##1}}}
\expandafter\def\csname PY@tok@ne\endcsname{\let\PY@bf=\textbf\def\PY@tc##1{\textcolor[rgb]{0.82,0.25,0.23}{##1}}}
\expandafter\def\csname PY@tok@nv\endcsname{\def\PY@tc##1{\textcolor[rgb]{0.10,0.09,0.49}{##1}}}
\expandafter\def\csname PY@tok@no\endcsname{\def\PY@tc##1{\textcolor[rgb]{0.53,0.00,0.00}{##1}}}
\expandafter\def\csname PY@tok@nl\endcsname{\def\PY@tc##1{\textcolor[rgb]{0.63,0.63,0.00}{##1}}}
\expandafter\def\csname PY@tok@ni\endcsname{\let\PY@bf=\textbf\def\PY@tc##1{\textcolor[rgb]{0.60,0.60,0.60}{##1}}}
\expandafter\def\csname PY@tok@na\endcsname{\def\PY@tc##1{\textcolor[rgb]{0.49,0.56,0.16}{##1}}}
\expandafter\def\csname PY@tok@nt\endcsname{\let\PY@bf=\textbf\def\PY@tc##1{\textcolor[rgb]{0.00,0.50,0.00}{##1}}}
\expandafter\def\csname PY@tok@nd\endcsname{\def\PY@tc##1{\textcolor[rgb]{0.67,0.13,1.00}{##1}}}
\expandafter\def\csname PY@tok@s\endcsname{\def\PY@tc##1{\textcolor[rgb]{0.73,0.13,0.13}{##1}}}
\expandafter\def\csname PY@tok@sd\endcsname{\let\PY@it=\textit\def\PY@tc##1{\textcolor[rgb]{0.73,0.13,0.13}{##1}}}
\expandafter\def\csname PY@tok@si\endcsname{\let\PY@bf=\textbf\def\PY@tc##1{\textcolor[rgb]{0.73,0.40,0.53}{##1}}}
\expandafter\def\csname PY@tok@se\endcsname{\let\PY@bf=\textbf\def\PY@tc##1{\textcolor[rgb]{0.73,0.40,0.13}{##1}}}
\expandafter\def\csname PY@tok@sr\endcsname{\def\PY@tc##1{\textcolor[rgb]{0.73,0.40,0.53}{##1}}}
\expandafter\def\csname PY@tok@ss\endcsname{\def\PY@tc##1{\textcolor[rgb]{0.10,0.09,0.49}{##1}}}
\expandafter\def\csname PY@tok@sx\endcsname{\def\PY@tc##1{\textcolor[rgb]{0.00,0.50,0.00}{##1}}}
\expandafter\def\csname PY@tok@m\endcsname{\def\PY@tc##1{\textcolor[rgb]{0.40,0.40,0.40}{##1}}}
\expandafter\def\csname PY@tok@gh\endcsname{\let\PY@bf=\textbf\def\PY@tc##1{\textcolor[rgb]{0.00,0.00,0.50}{##1}}}
\expandafter\def\csname PY@tok@gu\endcsname{\let\PY@bf=\textbf\def\PY@tc##1{\textcolor[rgb]{0.50,0.00,0.50}{##1}}}
\expandafter\def\csname PY@tok@gd\endcsname{\def\PY@tc##1{\textcolor[rgb]{0.63,0.00,0.00}{##1}}}
\expandafter\def\csname PY@tok@gi\endcsname{\def\PY@tc##1{\textcolor[rgb]{0.00,0.63,0.00}{##1}}}
\expandafter\def\csname PY@tok@gr\endcsname{\def\PY@tc##1{\textcolor[rgb]{1.00,0.00,0.00}{##1}}}
\expandafter\def\csname PY@tok@ge\endcsname{\let\PY@it=\textit}
\expandafter\def\csname PY@tok@gs\endcsname{\let\PY@bf=\textbf}
\expandafter\def\csname PY@tok@gp\endcsname{\let\PY@bf=\textbf\def\PY@tc##1{\textcolor[rgb]{0.00,0.00,0.50}{##1}}}
\expandafter\def\csname PY@tok@go\endcsname{\def\PY@tc##1{\textcolor[rgb]{0.53,0.53,0.53}{##1}}}
\expandafter\def\csname PY@tok@gt\endcsname{\def\PY@tc##1{\textcolor[rgb]{0.00,0.27,0.87}{##1}}}
\expandafter\def\csname PY@tok@err\endcsname{\def\PY@bc##1{\setlength{\fboxsep}{0pt}\fcolorbox[rgb]{1.00,0.00,0.00}{1,1,1}{\strut ##1}}}
\expandafter\def\csname PY@tok@kc\endcsname{\let\PY@bf=\textbf\def\PY@tc##1{\textcolor[rgb]{0.00,0.50,0.00}{##1}}}
\expandafter\def\csname PY@tok@kd\endcsname{\let\PY@bf=\textbf\def\PY@tc##1{\textcolor[rgb]{0.00,0.50,0.00}{##1}}}
\expandafter\def\csname PY@tok@kn\endcsname{\let\PY@bf=\textbf\def\PY@tc##1{\textcolor[rgb]{0.00,0.50,0.00}{##1}}}
\expandafter\def\csname PY@tok@kr\endcsname{\let\PY@bf=\textbf\def\PY@tc##1{\textcolor[rgb]{0.00,0.50,0.00}{##1}}}
\expandafter\def\csname PY@tok@bp\endcsname{\def\PY@tc##1{\textcolor[rgb]{0.00,0.50,0.00}{##1}}}
\expandafter\def\csname PY@tok@fm\endcsname{\def\PY@tc##1{\textcolor[rgb]{0.00,0.00,1.00}{##1}}}
\expandafter\def\csname PY@tok@vc\endcsname{\def\PY@tc##1{\textcolor[rgb]{0.10,0.09,0.49}{##1}}}
\expandafter\def\csname PY@tok@vg\endcsname{\def\PY@tc##1{\textcolor[rgb]{0.10,0.09,0.49}{##1}}}
\expandafter\def\csname PY@tok@vi\endcsname{\def\PY@tc##1{\textcolor[rgb]{0.10,0.09,0.49}{##1}}}
\expandafter\def\csname PY@tok@vm\endcsname{\def\PY@tc##1{\textcolor[rgb]{0.10,0.09,0.49}{##1}}}
\expandafter\def\csname PY@tok@sa\endcsname{\def\PY@tc##1{\textcolor[rgb]{0.73,0.13,0.13}{##1}}}
\expandafter\def\csname PY@tok@sb\endcsname{\def\PY@tc##1{\textcolor[rgb]{0.73,0.13,0.13}{##1}}}
\expandafter\def\csname PY@tok@sc\endcsname{\def\PY@tc##1{\textcolor[rgb]{0.73,0.13,0.13}{##1}}}
\expandafter\def\csname PY@tok@dl\endcsname{\def\PY@tc##1{\textcolor[rgb]{0.73,0.13,0.13}{##1}}}
\expandafter\def\csname PY@tok@s2\endcsname{\def\PY@tc##1{\textcolor[rgb]{0.73,0.13,0.13}{##1}}}
\expandafter\def\csname PY@tok@sh\endcsname{\def\PY@tc##1{\textcolor[rgb]{0.73,0.13,0.13}{##1}}}
\expandafter\def\csname PY@tok@s1\endcsname{\def\PY@tc##1{\textcolor[rgb]{0.73,0.13,0.13}{##1}}}
\expandafter\def\csname PY@tok@mb\endcsname{\def\PY@tc##1{\textcolor[rgb]{0.40,0.40,0.40}{##1}}}
\expandafter\def\csname PY@tok@mf\endcsname{\def\PY@tc##1{\textcolor[rgb]{0.40,0.40,0.40}{##1}}}
\expandafter\def\csname PY@tok@mh\endcsname{\def\PY@tc##1{\textcolor[rgb]{0.40,0.40,0.40}{##1}}}
\expandafter\def\csname PY@tok@mi\endcsname{\def\PY@tc##1{\textcolor[rgb]{0.40,0.40,0.40}{##1}}}
\expandafter\def\csname PY@tok@il\endcsname{\def\PY@tc##1{\textcolor[rgb]{0.40,0.40,0.40}{##1}}}
\expandafter\def\csname PY@tok@mo\endcsname{\def\PY@tc##1{\textcolor[rgb]{0.40,0.40,0.40}{##1}}}
\expandafter\def\csname PY@tok@ch\endcsname{\let\PY@it=\textit\def\PY@tc##1{\textcolor[rgb]{0.25,0.50,0.50}{##1}}}
\expandafter\def\csname PY@tok@cm\endcsname{\let\PY@it=\textit\def\PY@tc##1{\textcolor[rgb]{0.25,0.50,0.50}{##1}}}
\expandafter\def\csname PY@tok@cpf\endcsname{\let\PY@it=\textit\def\PY@tc##1{\textcolor[rgb]{0.25,0.50,0.50}{##1}}}
\expandafter\def\csname PY@tok@c1\endcsname{\let\PY@it=\textit\def\PY@tc##1{\textcolor[rgb]{0.25,0.50,0.50}{##1}}}
\expandafter\def\csname PY@tok@cs\endcsname{\let\PY@it=\textit\def\PY@tc##1{\textcolor[rgb]{0.25,0.50,0.50}{##1}}}

\def\PYZbs{\char`\\}
\def\PYZus{\char`\_}
\def\PYZob{\char`\{}
\def\PYZcb{\char`\}}
\def\PYZca{\char`\^}
\def\PYZam{\char`\&}
\def\PYZlt{\char`\<}
\def\PYZgt{\char`\>}
\def\PYZsh{\char`\#}
\def\PYZpc{\char`\%}
\def\PYZdl{\char`\$}
\def\PYZhy{\char`\-}
\def\PYZsq{\char`\'}
\def\PYZdq{\char`\"}
\def\PYZti{\char`\~}
% for compatibility with earlier versions
\def\PYZat{@}
\def\PYZlb{[}
\def\PYZrb{]}
\makeatother


    % Exact colors from NB
    \definecolor{incolor}{rgb}{0.0, 0.0, 0.5}
    \definecolor{outcolor}{rgb}{0.545, 0.0, 0.0}



    
    % Prevent overflowing lines due to hard-to-break entities
    \sloppy 
    % Setup hyperref package
    \hypersetup{
      breaklinks=true,  % so long urls are correctly broken across lines
      colorlinks=true,
      urlcolor=urlcolor,
      linkcolor=linkcolor,
      citecolor=citecolor,
      }
    % Slightly bigger margins than the latex defaults
    
    \geometry{verbose,tmargin=1in,bmargin=1in,lmargin=1in,rmargin=1in}
    
    

    \begin{document}
    
    
    \maketitle
    
    

    
    \hypertarget{analyze-ab-test-results}{%
\subsection{Analyze A/B Test Results}\label{analyze-ab-test-results}}

You may either submit your notebook through the workspace here, or you
may work from your local machine and submit through the next page.
Either way assure that your code passes the project
\href{https://review.udacity.com/\#!/projects/37e27304-ad47-4eb0-a1ab-8c12f60e43d0/rubric}{RUBRIC}.
**Please save regularly

This project will assure you have mastered the subjects covered in the
statistics lessons. The hope is to have this project be as comprehensive
of these topics as possible. Good luck!

\hypertarget{table-of-contents}{%
\subsection{Table of Contents}\label{table-of-contents}}

\begin{itemize}
\tightlist
\item
  Section \ref{intro}
\item
  Section \ref{probability}
\item
  Section \ref{ab_test}
\item
  Section \ref{regression}
\end{itemize}

 \#\#\# Introduction

A/B tests are very commonly performed by data analysts and data
scientists. It is important that you get some practice working with the
difficulties of these

For this project, you will be working to understand the results of an
A/B test run by an e-commerce website. Your goal is to work through this
notebook to help the company understand if they should implement the new
page, keep the old page, or perhaps run the experiment longer to make
their decision.

\textbf{As you work through this notebook, follow along in the classroom
and answer the corresponding quiz questions associated with each
question.} The labels for each classroom concept are provided for each
question. This will assure you are on the right track as you work
through the project, and you can feel more confident in your final
submission meeting the criteria. As a final check, assure you meet all
the criteria on the
\href{https://review.udacity.com/\#!/projects/37e27304-ad47-4eb0-a1ab-8c12f60e43d0/rubric}{RUBRIC}.

 \#\#\#\# Part I - Probability

To get started, let's import our libraries.

    \begin{Verbatim}[commandchars=\\\{\}]
{\color{incolor}In [{\color{incolor}46}]:} \PY{k+kn}{import} \PY{n+nn}{pandas} \PY{k}{as} \PY{n+nn}{pd}
         \PY{k+kn}{import} \PY{n+nn}{numpy} \PY{k}{as} \PY{n+nn}{np}
         \PY{k+kn}{import} \PY{n+nn}{random}
         \PY{k+kn}{import} \PY{n+nn}{matplotlib}\PY{n+nn}{.}\PY{n+nn}{pyplot} \PY{k}{as} \PY{n+nn}{plt}
         \PY{o}{\PYZpc{}}\PY{k}{matplotlib} inline
         \PY{c+c1}{\PYZsh{}We are setting the seed to assure you get the same answers on quizzes as we set up}
         \PY{n}{random}\PY{o}{.}\PY{n}{seed}\PY{p}{(}\PY{l+m+mi}{42}\PY{p}{)}
\end{Verbatim}


    \texttt{1.} Now, read in the \texttt{ab\_data.csv} data. Store it in
\texttt{df}. \textbf{Use your dataframe to answer the questions in Quiz
1 of the classroom.}

\begin{enumerate}
\def\labelenumi{\alph{enumi}.}
\tightlist
\item
  Read in the dataset and take a look at the top few rows here:
\end{enumerate}

    \begin{Verbatim}[commandchars=\\\{\}]
{\color{incolor}In [{\color{incolor}47}]:} \PY{n}{df} \PY{o}{=} \PY{n}{pd}\PY{o}{.}\PY{n}{read\PYZus{}csv}\PY{p}{(}\PY{l+s+s1}{\PYZsq{}}\PY{l+s+s1}{ab\PYZus{}data.csv}\PY{l+s+s1}{\PYZsq{}}\PY{p}{)}\PY{p}{;}
         \PY{n}{df}\PY{o}{.}\PY{n}{head}\PY{p}{(}\PY{p}{)}
\end{Verbatim}


\begin{Verbatim}[commandchars=\\\{\}]
{\color{outcolor}Out[{\color{outcolor}47}]:}    user\_id                   timestamp      group landing\_page  converted
         0   851104  2017-01-21 22:11:48.556739    control     old\_page          0
         1   804228  2017-01-12 08:01:45.159739    control     old\_page          0
         2   661590  2017-01-11 16:55:06.154213  treatment     new\_page          0
         3   853541  2017-01-08 18:28:03.143765  treatment     new\_page          0
         4   864975  2017-01-21 01:52:26.210827    control     old\_page          1
\end{Verbatim}
            
    \begin{enumerate}
\def\labelenumi{\alph{enumi}.}
\setcounter{enumi}{1}
\tightlist
\item
  Use the below cell to find the number of rows in the dataset.
\end{enumerate}

    \begin{Verbatim}[commandchars=\\\{\}]
{\color{incolor}In [{\color{incolor}48}]:} \PY{n}{numRows} \PY{o}{=} \PY{n}{df}\PY{o}{.}\PY{n}{shape}\PY{p}{[}\PY{l+m+mi}{0}\PY{p}{]}\PY{p}{;}
         \PY{n+nb}{print}\PY{p}{(}\PY{l+s+s1}{\PYZsq{}}\PY{l+s+s1}{Number of rows: }\PY{l+s+si}{\PYZob{}\PYZcb{}}\PY{l+s+s1}{\PYZsq{}}\PY{o}{.}\PY{n}{format}\PY{p}{(}\PY{n}{numRows}\PY{p}{)}\PY{p}{)}\PY{p}{;}
\end{Verbatim}


    \begin{Verbatim}[commandchars=\\\{\}]
Number of rows: 294478

    \end{Verbatim}

    \begin{enumerate}
\def\labelenumi{\alph{enumi}.}
\setcounter{enumi}{2}
\tightlist
\item
  The number of unique users in the dataset.
\end{enumerate}

    \begin{Verbatim}[commandchars=\\\{\}]
{\color{incolor}In [{\color{incolor}49}]:} \PY{n}{df\PYZus{}unique\PYZus{}users} \PY{o}{=} \PY{n}{df}\PY{o}{.}\PY{n}{groupby}\PY{p}{(}\PY{l+s+s1}{\PYZsq{}}\PY{l+s+s1}{user\PYZus{}id}\PY{l+s+s1}{\PYZsq{}}\PY{p}{,} \PY{n}{as\PYZus{}index}\PY{o}{=}\PY{k+kc}{False}\PY{p}{)}\PY{p}{;}
         \PY{n}{numUsers} \PY{o}{=} \PY{n+nb}{len}\PY{p}{(}\PY{n}{df\PYZus{}unique\PYZus{}users}\PY{p}{)}\PY{p}{;}
         \PY{n+nb}{print}\PY{p}{(}\PY{l+s+s1}{\PYZsq{}}\PY{l+s+s1}{Number of unique users: }\PY{l+s+si}{\PYZob{}\PYZcb{}}\PY{l+s+s1}{\PYZsq{}}\PY{o}{.}\PY{n}{format}\PY{p}{(}\PY{n}{numUsers}\PY{p}{)}\PY{p}{)}\PY{p}{;}
\end{Verbatim}


    \begin{Verbatim}[commandchars=\\\{\}]
Number of unique users: 290584

    \end{Verbatim}

    \begin{enumerate}
\def\labelenumi{\alph{enumi}.}
\setcounter{enumi}{3}
\tightlist
\item
  The proportion of users converted.
\end{enumerate}

    \begin{Verbatim}[commandchars=\\\{\}]
{\color{incolor}In [{\color{incolor}50}]:} \PY{n}{numUsersConverted} \PY{o}{=} \PY{n}{df\PYZus{}unique\PYZus{}users}\PY{o}{.}\PY{n}{sum}\PY{p}{(}\PY{p}{)}\PY{o}{.}\PY{n}{converted}\PY{o}{.}\PY{n}{sum}\PY{p}{(}\PY{p}{)}\PY{p}{;}
         \PY{n+nb}{print}\PY{p}{(}\PY{l+s+s1}{\PYZsq{}}\PY{l+s+s1}{Number of users converted: }\PY{l+s+si}{\PYZob{}\PYZcb{}}\PY{l+s+s1}{\PYZsq{}}\PY{o}{.}\PY{n}{format}\PY{p}{(}\PY{n}{numUsersConverted}\PY{p}{)}\PY{p}{)}\PY{p}{;}
         
         \PY{n}{p\PYZus{}usersConverted} \PY{o}{=} \PY{n}{numUsersConverted}\PY{o}{/}\PY{n}{numUsers}\PY{p}{;}
         \PY{n+nb}{print}\PY{p}{(}\PY{l+s+s1}{\PYZsq{}}\PY{l+s+s1}{Proportion of users converted: }\PY{l+s+si}{\PYZob{}\PYZcb{}}\PY{l+s+s1}{\PYZsq{}}\PY{o}{.}\PY{n}{format}\PY{p}{(}\PY{n}{p\PYZus{}usersConverted}\PY{p}{)}\PY{p}{)}\PY{p}{;}
\end{Verbatim}


    \begin{Verbatim}[commandchars=\\\{\}]
Number of users converted: 35237
Proportion of users converted: 0.12126269856564711

    \end{Verbatim}

    \begin{enumerate}
\def\labelenumi{\alph{enumi}.}
\setcounter{enumi}{4}
\tightlist
\item
  The number of times the \texttt{new\_page} and \texttt{treatment}
  don't line up.
\end{enumerate}

    \begin{Verbatim}[commandchars=\\\{\}]
{\color{incolor}In [{\color{incolor}51}]:} \PY{n}{numControlMismatch} \PY{o}{=} \PY{n}{df}\PY{o}{.}\PY{n}{query}\PY{p}{(}\PY{l+s+s1}{\PYZsq{}}\PY{l+s+s1}{((group == }\PY{l+s+s1}{\PYZdq{}}\PY{l+s+s1}{control}\PY{l+s+s1}{\PYZdq{}}\PY{l+s+s1}{) \PYZam{} (landing\PYZus{}page == }\PY{l+s+s1}{\PYZdq{}}\PY{l+s+s1}{new\PYZus{}page}\PY{l+s+s1}{\PYZdq{}}\PY{l+s+s1}{))}\PY{l+s+s1}{\PYZsq{}}\PY{p}{)}\PY{o}{.}\PY{n}{shape}\PY{p}{[}\PY{l+m+mi}{0}\PY{p}{]}\PY{p}{;}
         \PY{n}{numTreatmentMismatch} \PY{o}{=} \PY{n}{df}\PY{o}{.}\PY{n}{query}\PY{p}{(}\PY{l+s+s1}{\PYZsq{}}\PY{l+s+s1}{((group == }\PY{l+s+s1}{\PYZdq{}}\PY{l+s+s1}{treatment}\PY{l+s+s1}{\PYZdq{}}\PY{l+s+s1}{) \PYZam{} (landing\PYZus{}page == }\PY{l+s+s1}{\PYZdq{}}\PY{l+s+s1}{old\PYZus{}page}\PY{l+s+s1}{\PYZdq{}}\PY{l+s+s1}{))}\PY{l+s+s1}{\PYZsq{}}\PY{p}{)}\PY{o}{.}\PY{n}{shape}\PY{p}{[}\PY{l+m+mi}{0}\PY{p}{]}\PY{p}{;}
         
         \PY{n}{numDoNotLineUp} \PY{o}{=} \PY{n}{numControlMismatch} \PY{o}{+} \PY{n}{numTreatmentMismatch}\PY{p}{;}
         
         \PY{n+nb}{print}\PY{p}{(}\PY{l+s+s1}{\PYZsq{}}\PY{l+s+s1}{Number of times the new\PYZus{}page and treatment do not line up: }\PY{l+s+si}{\PYZob{}\PYZcb{}}\PY{l+s+s1}{\PYZsq{}}\PY{o}{.}\PY{n}{format}\PY{p}{(}\PY{n}{numDoNotLineUp}\PY{p}{)}\PY{p}{)}\PY{p}{;}
\end{Verbatim}


    \begin{Verbatim}[commandchars=\\\{\}]
Number of times the new\_page and treatment do not line up: 3893

    \end{Verbatim}

    \begin{enumerate}
\def\labelenumi{\alph{enumi}.}
\setcounter{enumi}{5}
\tightlist
\item
  Do any of the rows have missing values?
\end{enumerate}

    \begin{Verbatim}[commandchars=\\\{\}]
{\color{incolor}In [{\color{incolor}52}]:} \PY{c+c1}{\PYZsh{} There are no rows with missing values}
         \PY{n}{df}\PY{o}{.}\PY{n}{isnull}\PY{p}{(}\PY{p}{)}\PY{o}{.}\PY{n}{values}\PY{o}{.}\PY{n}{any}\PY{p}{(}\PY{p}{)}
\end{Verbatim}


\begin{Verbatim}[commandchars=\\\{\}]
{\color{outcolor}Out[{\color{outcolor}52}]:} False
\end{Verbatim}
            
    \texttt{2.} For the rows where \textbf{treatment} is not aligned with
\textbf{new\_page} or \textbf{control} is not aligned with
\textbf{old\_page}, we cannot be sure if this row truly received the new
or old page. Use \textbf{Quiz 2} in the classroom to provide how we
should handle these rows.

\begin{enumerate}
\def\labelenumi{\alph{enumi}.}
\tightlist
\item
  Now use the answer to the quiz to create a new dataset that meets the
  specifications from the quiz. Store your new dataframe in
  \textbf{df2}.
\end{enumerate}

    \begin{Verbatim}[commandchars=\\\{\}]
{\color{incolor}In [{\color{incolor}53}]:} \PY{n}{df2} \PY{o}{=} \PY{n}{df}\PY{o}{.}\PY{n}{query}\PY{p}{(}\PY{l+s+s1}{\PYZsq{}}\PY{l+s+s1}{(((group == }\PY{l+s+s1}{\PYZdq{}}\PY{l+s+s1}{control}\PY{l+s+s1}{\PYZdq{}}\PY{l+s+s1}{) \PYZam{} (landing\PYZus{}page == }\PY{l+s+s1}{\PYZdq{}}\PY{l+s+s1}{old\PYZus{}page}\PY{l+s+s1}{\PYZdq{}}\PY{l+s+s1}{))) | (((group == }\PY{l+s+s1}{\PYZdq{}}\PY{l+s+s1}{treatment}\PY{l+s+s1}{\PYZdq{}}\PY{l+s+s1}{) \PYZam{} (landing\PYZus{}page == }\PY{l+s+s1}{\PYZdq{}}\PY{l+s+s1}{new\PYZus{}page}\PY{l+s+s1}{\PYZdq{}}\PY{l+s+s1}{))) }\PY{l+s+s1}{\PYZsq{}}\PY{p}{)}\PY{p}{;}
         \PY{n}{df2}\PY{o}{.}\PY{n}{head}\PY{p}{(}\PY{p}{)}
\end{Verbatim}


\begin{Verbatim}[commandchars=\\\{\}]
{\color{outcolor}Out[{\color{outcolor}53}]:}    user\_id                   timestamp      group landing\_page  converted
         0   851104  2017-01-21 22:11:48.556739    control     old\_page          0
         1   804228  2017-01-12 08:01:45.159739    control     old\_page          0
         2   661590  2017-01-11 16:55:06.154213  treatment     new\_page          0
         3   853541  2017-01-08 18:28:03.143765  treatment     new\_page          0
         4   864975  2017-01-21 01:52:26.210827    control     old\_page          1
\end{Verbatim}
            
    \begin{Verbatim}[commandchars=\\\{\}]
{\color{incolor}In [{\color{incolor}54}]:} \PY{c+c1}{\PYZsh{} Double Check all of the correct rows were removed \PYZhy{} this should be 0}
         \PY{n}{df2}\PY{p}{[}\PY{p}{(}\PY{p}{(}\PY{n}{df2}\PY{p}{[}\PY{l+s+s1}{\PYZsq{}}\PY{l+s+s1}{group}\PY{l+s+s1}{\PYZsq{}}\PY{p}{]} \PY{o}{==} \PY{l+s+s1}{\PYZsq{}}\PY{l+s+s1}{treatment}\PY{l+s+s1}{\PYZsq{}}\PY{p}{)} \PY{o}{==} \PY{p}{(}\PY{n}{df2}\PY{p}{[}\PY{l+s+s1}{\PYZsq{}}\PY{l+s+s1}{landing\PYZus{}page}\PY{l+s+s1}{\PYZsq{}}\PY{p}{]} \PY{o}{==} \PY{l+s+s1}{\PYZsq{}}\PY{l+s+s1}{new\PYZus{}page}\PY{l+s+s1}{\PYZsq{}}\PY{p}{)}\PY{p}{)} \PY{o}{==} \PY{k+kc}{False}\PY{p}{]}\PY{o}{.}\PY{n}{shape}\PY{p}{[}\PY{l+m+mi}{0}\PY{p}{]}
\end{Verbatim}


\begin{Verbatim}[commandchars=\\\{\}]
{\color{outcolor}Out[{\color{outcolor}54}]:} 0
\end{Verbatim}
            
    \texttt{3.} Use \textbf{df2} and the cells below to answer questions for
\textbf{Quiz3} in the classroom.

    \begin{enumerate}
\def\labelenumi{\alph{enumi}.}
\tightlist
\item
  How many unique \textbf{user\_id}s are in \textbf{df2}?
\end{enumerate}

    \begin{Verbatim}[commandchars=\\\{\}]
{\color{incolor}In [{\color{incolor}55}]:} \PY{n}{df2\PYZus{}unique\PYZus{}users} \PY{o}{=} \PY{n}{df2}\PY{o}{.}\PY{n}{groupby}\PY{p}{(}\PY{l+s+s1}{\PYZsq{}}\PY{l+s+s1}{user\PYZus{}id}\PY{l+s+s1}{\PYZsq{}}\PY{p}{)}\PY{p}{;}
         \PY{n}{numUsers2} \PY{o}{=} \PY{n+nb}{len}\PY{p}{(}\PY{n}{df2\PYZus{}unique\PYZus{}users}\PY{p}{)}\PY{p}{;}
         
         \PY{n+nb}{print}\PY{p}{(}\PY{l+s+s1}{\PYZsq{}}\PY{l+s+s1}{Number of unique users: }\PY{l+s+si}{\PYZob{}\PYZcb{}}\PY{l+s+s1}{\PYZsq{}}\PY{o}{.}\PY{n}{format}\PY{p}{(}\PY{n}{numUsers2}\PY{p}{)}\PY{p}{)}\PY{p}{;}
\end{Verbatim}


    \begin{Verbatim}[commandchars=\\\{\}]
Number of unique users: 290584

    \end{Verbatim}

    \begin{enumerate}
\def\labelenumi{\alph{enumi}.}
\setcounter{enumi}{1}
\tightlist
\item
  There is one \textbf{user\_id} repeated in \textbf{df2}. What is it?
\end{enumerate}

    \begin{Verbatim}[commandchars=\\\{\}]
{\color{incolor}In [{\color{incolor}56}]:} \PY{n}{df\PYZus{}user\PYZus{}id} \PY{o}{=} \PY{n}{df2}\PY{p}{[}\PY{l+s+s1}{\PYZsq{}}\PY{l+s+s1}{user\PYZus{}id}\PY{l+s+s1}{\PYZsq{}}\PY{p}{]}\PY{p}{;}
         \PY{n}{df\PYZus{}duplicated\PYZus{}ids} \PY{o}{=} \PY{n}{df\PYZus{}user\PYZus{}id}\PY{p}{[}\PY{n}{df\PYZus{}user\PYZus{}id}\PY{o}{.}\PY{n}{duplicated}\PY{p}{(}\PY{n}{keep}\PY{o}{=}\PY{k+kc}{False}\PY{p}{)}\PY{p}{]}\PY{p}{;}
         
         \PY{n}{duplicated\PYZus{}id} \PY{o}{=} \PY{n}{df\PYZus{}duplicated\PYZus{}ids}\PY{o}{.}\PY{n}{iloc}\PY{p}{[}\PY{l+m+mi}{0}\PY{p}{]}\PY{p}{;}
         \PY{n+nb}{print}\PY{p}{(}\PY{l+s+s1}{\PYZsq{}}\PY{l+s+s1}{The duplicated user\PYZus{}id is }\PY{l+s+si}{\PYZob{}\PYZcb{}}\PY{l+s+s1}{\PYZsq{}}\PY{o}{.}\PY{n}{format}\PY{p}{(}\PY{n}{duplicated\PYZus{}id}\PY{p}{)}\PY{p}{)}\PY{p}{;}
\end{Verbatim}


    \begin{Verbatim}[commandchars=\\\{\}]
The duplicated user\_id is 773192

    \end{Verbatim}

    \begin{enumerate}
\def\labelenumi{\alph{enumi}.}
\setcounter{enumi}{2}
\tightlist
\item
  What is the row information for the repeat \textbf{user\_id}?
\end{enumerate}

    \begin{Verbatim}[commandchars=\\\{\}]
{\color{incolor}In [{\color{incolor}57}]:} \PY{n}{query\PYZus{}duped\PYZus{}id} \PY{o}{=} \PY{l+s+s1}{\PYZsq{}}\PY{l+s+s1}{user\PYZus{}id == }\PY{l+s+s1}{\PYZsq{}} \PY{o}{+} \PY{n+nb}{str}\PY{p}{(}\PY{n}{duplicated\PYZus{}id}\PY{p}{)}\PY{p}{;}
         \PY{n}{df2}\PY{o}{.}\PY{n}{query}\PY{p}{(}\PY{n}{query\PYZus{}duped\PYZus{}id}\PY{p}{)}
\end{Verbatim}


\begin{Verbatim}[commandchars=\\\{\}]
{\color{outcolor}Out[{\color{outcolor}57}]:}       user\_id                   timestamp      group landing\_page  converted
         1899   773192  2017-01-09 05:37:58.781806  treatment     new\_page          0
         2893   773192  2017-01-14 02:55:59.590927  treatment     new\_page          0
\end{Verbatim}
            
    \begin{enumerate}
\def\labelenumi{\alph{enumi}.}
\setcounter{enumi}{3}
\tightlist
\item
  Remove \textbf{one} of the rows with a duplicate \textbf{user\_id},
  but keep your dataframe as \textbf{df2}.
\end{enumerate}

    \begin{Verbatim}[commandchars=\\\{\}]
{\color{incolor}In [{\color{incolor}58}]:} \PY{c+c1}{\PYZsh{} Drop the second row of the duplicate user\PYZus{}id}
         \PY{n}{df2} \PY{o}{=} \PY{n}{df2}\PY{o}{.}\PY{n}{drop}\PY{p}{(}\PY{n}{df2}\PY{o}{.}\PY{n}{query}\PY{p}{(}\PY{n}{query\PYZus{}duped\PYZus{}id}\PY{p}{)}\PY{o}{.}\PY{n}{index}\PY{p}{[}\PY{l+m+mi}{1}\PY{p}{]}\PY{p}{)}\PY{p}{;}
\end{Verbatim}


    \begin{Verbatim}[commandchars=\\\{\}]
{\color{incolor}In [{\color{incolor}59}]:} \PY{c+c1}{\PYZsh{} Check to ensure that the duplicated ids are removed}
         \PY{n}{df\PYZus{}user\PYZus{}id} \PY{o}{=} \PY{n}{df2}\PY{p}{[}\PY{l+s+s1}{\PYZsq{}}\PY{l+s+s1}{user\PYZus{}id}\PY{l+s+s1}{\PYZsq{}}\PY{p}{]}\PY{p}{;}
         \PY{n}{df\PYZus{}user\PYZus{}id}\PY{p}{[}\PY{n}{df\PYZus{}user\PYZus{}id}\PY{o}{.}\PY{n}{duplicated}\PY{p}{(}\PY{n}{keep}\PY{o}{=}\PY{k+kc}{False}\PY{p}{)}\PY{p}{]}
\end{Verbatim}


\begin{Verbatim}[commandchars=\\\{\}]
{\color{outcolor}Out[{\color{outcolor}59}]:} Series([], Name: user\_id, dtype: int64)
\end{Verbatim}
            
    \texttt{4.} Use \textbf{df2} in the below cells to answer the quiz
questions related to \textbf{Quiz 4} in the classroom.

\begin{enumerate}
\def\labelenumi{\alph{enumi}.}
\tightlist
\item
  What is the probability of an individual converting regardless of the
  page they receive?
\end{enumerate}

    \begin{Verbatim}[commandchars=\\\{\}]
{\color{incolor}In [{\color{incolor}60}]:} \PY{n}{p\PYZus{}convert} \PY{o}{=} \PY{n}{df2}\PY{p}{[}\PY{l+s+s1}{\PYZsq{}}\PY{l+s+s1}{converted}\PY{l+s+s1}{\PYZsq{}}\PY{p}{]}\PY{o}{.}\PY{n}{mean}\PY{p}{(}\PY{p}{)}\PY{p}{;}
         
         \PY{n+nb}{print}\PY{p}{(}\PY{l+s+s1}{\PYZsq{}}\PY{l+s+s1}{Probablity of converting regardless of page: }\PY{l+s+si}{\PYZob{}\PYZcb{}}\PY{l+s+s1}{\PYZsq{}}\PY{o}{.}\PY{n}{format}\PY{p}{(}\PY{n}{p\PYZus{}convert}\PY{p}{)}\PY{p}{)}\PY{p}{;}
\end{Verbatim}


    \begin{Verbatim}[commandchars=\\\{\}]
Probablity of converting regardless of page: 0.11959708724499628

    \end{Verbatim}

    \begin{enumerate}
\def\labelenumi{\alph{enumi}.}
\setcounter{enumi}{1}
\tightlist
\item
  Given that an individual was in the \texttt{control} group, what is
  the probability they converted?
\end{enumerate}

    \begin{Verbatim}[commandchars=\\\{\}]
{\color{incolor}In [{\color{incolor}61}]:} \PY{n}{df2\PYZus{}control} \PY{o}{=} \PY{n}{df2}\PY{o}{.}\PY{n}{query}\PY{p}{(}\PY{l+s+s1}{\PYZsq{}}\PY{l+s+s1}{group == }\PY{l+s+s1}{\PYZdq{}}\PY{l+s+s1}{control}\PY{l+s+s1}{\PYZdq{}}\PY{l+s+s1}{\PYZsq{}}\PY{p}{)}\PY{p}{;}
         
         \PY{n}{p\PYZus{}convert\PYZus{}control} \PY{o}{=} \PY{n}{df2\PYZus{}control}\PY{p}{[}\PY{l+s+s1}{\PYZsq{}}\PY{l+s+s1}{converted}\PY{l+s+s1}{\PYZsq{}}\PY{p}{]}\PY{o}{.}\PY{n}{mean}\PY{p}{(}\PY{p}{)}\PY{p}{;}
         \PY{n+nb}{print}\PY{p}{(}\PY{l+s+s1}{\PYZsq{}}\PY{l+s+s1}{Probablity of converting given control group: }\PY{l+s+si}{\PYZob{}\PYZcb{}}\PY{l+s+s1}{\PYZsq{}}\PY{o}{.}\PY{n}{format}\PY{p}{(}\PY{n}{p\PYZus{}convert\PYZus{}control}\PY{p}{)}\PY{p}{)}\PY{p}{;}
\end{Verbatim}


    \begin{Verbatim}[commandchars=\\\{\}]
Probablity of converting given control group: 0.1203863045004612

    \end{Verbatim}

    \begin{enumerate}
\def\labelenumi{\alph{enumi}.}
\setcounter{enumi}{2}
\tightlist
\item
  Given that an individual was in the \texttt{treatment} group, what is
  the probability they converted?
\end{enumerate}

    \begin{Verbatim}[commandchars=\\\{\}]
{\color{incolor}In [{\color{incolor}62}]:} \PY{n}{df2\PYZus{}treatment} \PY{o}{=} \PY{n}{df2}\PY{o}{.}\PY{n}{query}\PY{p}{(}\PY{l+s+s1}{\PYZsq{}}\PY{l+s+s1}{group == }\PY{l+s+s1}{\PYZdq{}}\PY{l+s+s1}{treatment}\PY{l+s+s1}{\PYZdq{}}\PY{l+s+s1}{\PYZsq{}}\PY{p}{)}\PY{p}{;}
         
         \PY{n}{p\PYZus{}convert\PYZus{}treatment} \PY{o}{=} \PY{n}{df2\PYZus{}treatment}\PY{p}{[}\PY{l+s+s1}{\PYZsq{}}\PY{l+s+s1}{converted}\PY{l+s+s1}{\PYZsq{}}\PY{p}{]}\PY{o}{.}\PY{n}{mean}\PY{p}{(}\PY{p}{)}\PY{p}{;}
         \PY{n+nb}{print}\PY{p}{(}\PY{l+s+s1}{\PYZsq{}}\PY{l+s+s1}{Probablity of converting given treatment group: }\PY{l+s+si}{\PYZob{}\PYZcb{}}\PY{l+s+s1}{\PYZsq{}}\PY{o}{.}\PY{n}{format}\PY{p}{(}\PY{n}{p\PYZus{}convert\PYZus{}treatment}\PY{p}{)}\PY{p}{)}\PY{p}{;}
\end{Verbatim}


    \begin{Verbatim}[commandchars=\\\{\}]
Probablity of converting given treatment group: 0.11880806551510564

    \end{Verbatim}

    \begin{enumerate}
\def\labelenumi{\alph{enumi}.}
\setcounter{enumi}{3}
\tightlist
\item
  What is the probability that an individual received the new page?
\end{enumerate}

    \begin{Verbatim}[commandchars=\\\{\}]
{\color{incolor}In [{\color{incolor}63}]:} \PY{n}{num\PYZus{}new\PYZus{}page} \PY{o}{=} \PY{n}{df2}\PY{o}{.}\PY{n}{query}\PY{p}{(}\PY{l+s+s1}{\PYZsq{}}\PY{l+s+s1}{landing\PYZus{}page == }\PY{l+s+s1}{\PYZdq{}}\PY{l+s+s1}{new\PYZus{}page}\PY{l+s+s1}{\PYZdq{}}\PY{l+s+s1}{\PYZsq{}}\PY{p}{)}\PY{o}{.}\PY{n}{shape}\PY{p}{[}\PY{l+m+mi}{0}\PY{p}{]}\PY{p}{;}
         \PY{n}{num\PYZus{}df2} \PY{o}{=} \PY{n}{df2}\PY{o}{.}\PY{n}{shape}\PY{p}{[}\PY{l+m+mi}{0}\PY{p}{]}\PY{p}{;}
         
         \PY{n}{p\PYZus{}new\PYZus{}page} \PY{o}{=} \PY{n}{num\PYZus{}new\PYZus{}page} \PY{o}{/} \PY{n}{num\PYZus{}df2}\PY{p}{;}
         \PY{n+nb}{print}\PY{p}{(}\PY{l+s+s1}{\PYZsq{}}\PY{l+s+s1}{Probability of receiving the new page: }\PY{l+s+si}{\PYZob{}\PYZcb{}}\PY{l+s+s1}{\PYZsq{}}\PY{o}{.}\PY{n}{format}\PY{p}{(}\PY{n}{p\PYZus{}new\PYZus{}page}\PY{p}{)}\PY{p}{)}\PY{p}{;}
\end{Verbatim}


    \begin{Verbatim}[commandchars=\\\{\}]
Probability of receiving the new page: 0.5000619442226688

    \end{Verbatim}

    \begin{enumerate}
\def\labelenumi{\alph{enumi}.}
\setcounter{enumi}{4}
\tightlist
\item
  Use the results in the previous two portions of this question to
  suggest if you think there is evidence that one page leads to more
  conversions? Write your response below.
\end{enumerate}

    \textbf{Answer:}

From the probability of converting for each individual page that we
found above, there is no clear evidence that either page will lead to
more conversion.

     \#\#\# Part II - A/B Test

Notice that because of the time stamp associated with each event, you
could technically run a hypothesis test continuously as each observation
was observed.

However, then the hard question is do you stop as soon as one page is
considered significantly better than another or does it need to happen
consistently for a certain amount of time? How long do you run to render
a decision that neither page is better than another?

These questions are the difficult parts associated with A/B tests in
general.

\texttt{1.} For now, consider you need to make the decision just based
on all the data provided. If you want to assume that the old page is
better unless the new page proves to be definitely better at a Type I
error rate of 5\%, what should your null and alternative hypotheses be?
You can state your hypothesis in terms of words or in terms of
\textbf{\(p_{old}\)} and \textbf{\(p_{new}\)}, which are the converted
rates for the old and new pages.

    \[H_0: p_{new} - p_{old} \leq 0\]

\[H_1: p_{new} - p_{old} > 0\]

\textbf{\(p_{new}\) and \(p_{old}\) are the converted rates for the old
and new pages, respectivley.}

    \texttt{2.} Assume under the null hypothesis, \(p_{new}\) and
\(p_{old}\) both have ``true'' success rates equal to the
\textbf{converted} success rate regardless of page - that is \(p_{new}\)
and \(p_{old}\) are equal. Furthermore, assume they are equal to the
\textbf{converted} rate in \textbf{ab\_data.csv} regardless of the page.

Use a sample size for each page equal to the ones in
\textbf{ab\_data.csv}.

Perform the sampling distribution for the difference in
\textbf{converted} between the two pages over 10,000 iterations of
calculating an estimate from the null.

Use the cells below to provide the necessary parts of this simulation.
If this doesn't make complete sense right now, don't worry - you are
going to work through the problems below to complete this problem. You
can use \textbf{Quiz 5} in the classroom to make sure you are on the
right track.

    \begin{enumerate}
\def\labelenumi{\alph{enumi}.}
\tightlist
\item
  What is the \textbf{convert rate} for \(p_{new}\) under the null?
\end{enumerate}

    \begin{Verbatim}[commandchars=\\\{\}]
{\color{incolor}In [{\color{incolor}64}]:} \PY{c+c1}{\PYZsh{} Under the null hypothesis, we assume that p\PYZus{}new and p\PYZus{}old }
         \PY{c+c1}{\PYZsh{} are equal to the converted rate in ab\PYZus{}data.csv.}
         
         \PY{c+c1}{\PYZsh{} Hence, p\PYZus{}new can be found from the converted rate of the data}
         
         \PY{n}{p\PYZus{}new} \PY{o}{=} \PY{n}{df}\PY{p}{[}\PY{l+s+s1}{\PYZsq{}}\PY{l+s+s1}{converted}\PY{l+s+s1}{\PYZsq{}}\PY{p}{]}\PY{o}{.}\PY{n}{mean}\PY{p}{(}\PY{p}{)}\PY{p}{;}
         \PY{n+nb}{print}\PY{p}{(}\PY{l+s+s1}{\PYZsq{}}\PY{l+s+s1}{p\PYZus{}new: }\PY{l+s+si}{\PYZob{}\PYZcb{}}\PY{l+s+s1}{\PYZsq{}}\PY{o}{.}\PY{n}{format}\PY{p}{(}\PY{n}{p\PYZus{}new}\PY{p}{)}\PY{p}{)}\PY{p}{;}
\end{Verbatim}


    \begin{Verbatim}[commandchars=\\\{\}]
p\_new: 0.11965919355605512

    \end{Verbatim}

    \begin{enumerate}
\def\labelenumi{\alph{enumi}.}
\setcounter{enumi}{1}
\tightlist
\item
  What is the \textbf{convert rate} for \(p_{old}\) under the null? 
\end{enumerate}

    \begin{Verbatim}[commandchars=\\\{\}]
{\color{incolor}In [{\color{incolor}65}]:} \PY{c+c1}{\PYZsh{} Under the null hypothesis, we assume that p\PYZus{}old = p\PYZus{}new.}
         
         \PY{n}{p\PYZus{}old} \PY{o}{=} \PY{n}{p\PYZus{}new}\PY{p}{;}
         \PY{n+nb}{print}\PY{p}{(}\PY{l+s+s1}{\PYZsq{}}\PY{l+s+s1}{p\PYZus{}old: }\PY{l+s+si}{\PYZob{}\PYZcb{}}\PY{l+s+s1}{\PYZsq{}}\PY{o}{.}\PY{n}{format}\PY{p}{(}\PY{n}{p\PYZus{}old}\PY{p}{)}\PY{p}{)}\PY{p}{;}
\end{Verbatim}


    \begin{Verbatim}[commandchars=\\\{\}]
p\_old: 0.11965919355605512

    \end{Verbatim}

    \begin{enumerate}
\def\labelenumi{\alph{enumi}.}
\setcounter{enumi}{2}
\tightlist
\item
  What is \(n_{new}\)?
\end{enumerate}

    \begin{Verbatim}[commandchars=\\\{\}]
{\color{incolor}In [{\color{incolor}66}]:} \PY{c+c1}{\PYZsh{} Under the null hypothesis, we assume that a sample size for }
         \PY{c+c1}{\PYZsh{} each page equal to the ones in ab\PYZus{}data.csv,}
         
         \PY{c+c1}{\PYZsh{} Hence, we can find n\PYZus{}new from a number of rows in group==\PYZsq{}treatment\PYZsq{}}
         
         \PY{n}{n\PYZus{}new} \PY{o}{=} \PY{n}{df2}\PY{o}{.}\PY{n}{query}\PY{p}{(}\PY{l+s+s1}{\PYZsq{}}\PY{l+s+s1}{group == }\PY{l+s+s1}{\PYZdq{}}\PY{l+s+s1}{treatment}\PY{l+s+s1}{\PYZdq{}}\PY{l+s+s1}{\PYZsq{}}\PY{p}{)}\PY{o}{.}\PY{n}{shape}\PY{p}{[}\PY{l+m+mi}{0}\PY{p}{]}\PY{p}{;}
         
         \PY{n+nb}{print}\PY{p}{(}\PY{l+s+s1}{\PYZsq{}}\PY{l+s+s1}{n\PYZus{}new: }\PY{l+s+si}{\PYZob{}\PYZcb{}}\PY{l+s+s1}{\PYZsq{}}\PY{o}{.}\PY{n}{format}\PY{p}{(}\PY{n}{n\PYZus{}new}\PY{p}{)}\PY{p}{)}\PY{p}{;}
\end{Verbatim}


    \begin{Verbatim}[commandchars=\\\{\}]
n\_new: 145310

    \end{Verbatim}

    \begin{enumerate}
\def\labelenumi{\alph{enumi}.}
\setcounter{enumi}{3}
\tightlist
\item
  What is \(n_{old}\)?
\end{enumerate}

    \begin{Verbatim}[commandchars=\\\{\}]
{\color{incolor}In [{\color{incolor}67}]:} \PY{c+c1}{\PYZsh{} Also, we can find n\PYZus{}old from a number of rows in group==\PYZsq{}control\PYZsq{}}
         \PY{n}{n\PYZus{}old} \PY{o}{=} \PY{n}{df2}\PY{o}{.}\PY{n}{query}\PY{p}{(}\PY{l+s+s1}{\PYZsq{}}\PY{l+s+s1}{group == }\PY{l+s+s1}{\PYZdq{}}\PY{l+s+s1}{control}\PY{l+s+s1}{\PYZdq{}}\PY{l+s+s1}{\PYZsq{}}\PY{p}{)}\PY{o}{.}\PY{n}{shape}\PY{p}{[}\PY{l+m+mi}{0}\PY{p}{]}\PY{p}{;}
         
         \PY{n+nb}{print}\PY{p}{(}\PY{l+s+s1}{\PYZsq{}}\PY{l+s+s1}{n\PYZus{}old: }\PY{l+s+si}{\PYZob{}\PYZcb{}}\PY{l+s+s1}{\PYZsq{}}\PY{o}{.}\PY{n}{format}\PY{p}{(}\PY{n}{n\PYZus{}old}\PY{p}{)}\PY{p}{)}\PY{p}{;}
\end{Verbatim}


    \begin{Verbatim}[commandchars=\\\{\}]
n\_old: 145274

    \end{Verbatim}

    \begin{enumerate}
\def\labelenumi{\alph{enumi}.}
\setcounter{enumi}{4}
\tightlist
\item
  Simulate \(n_{new}\) transactions with a convert rate of \(p_{new}\)
  under the null. Store these \(n_{new}\) 1's and 0's in
  \textbf{new\_page\_converted}.
\end{enumerate}

    \begin{Verbatim}[commandchars=\\\{\}]
{\color{incolor}In [{\color{incolor}68}]:} \PY{n}{new\PYZus{}page\PYZus{}converted} \PY{o}{=} \PY{n}{np}\PY{o}{.}\PY{n}{random}\PY{o}{.}\PY{n}{choice}\PY{p}{(}\PY{p}{[}\PY{l+m+mi}{0}\PY{p}{,} \PY{l+m+mi}{1}\PY{p}{]}\PY{p}{,} \PY{n}{n\PYZus{}new}\PY{p}{,} \PY{n}{p}\PY{o}{=}\PY{p}{[}\PY{l+m+mi}{1}\PY{o}{\PYZhy{}}\PY{n}{p\PYZus{}new}\PY{p}{,} \PY{n}{p\PYZus{}new}\PY{p}{]}\PY{p}{)}\PY{p}{;}
\end{Verbatim}


    \begin{enumerate}
\def\labelenumi{\alph{enumi}.}
\setcounter{enumi}{5}
\tightlist
\item
  Simulate \(n_{old}\) transactions with a convert rate of \(p_{old}\)
  under the null. Store these \(n_{old}\) 1's and 0's in
  \textbf{old\_page\_converted}.
\end{enumerate}

    \begin{Verbatim}[commandchars=\\\{\}]
{\color{incolor}In [{\color{incolor}69}]:} \PY{n}{old\PYZus{}page\PYZus{}converted} \PY{o}{=} \PY{n}{np}\PY{o}{.}\PY{n}{random}\PY{o}{.}\PY{n}{choice}\PY{p}{(}\PY{p}{[}\PY{l+m+mi}{0}\PY{p}{,} \PY{l+m+mi}{1}\PY{p}{]}\PY{p}{,} \PY{n}{n\PYZus{}old}\PY{p}{,} \PY{n}{p}\PY{o}{=}\PY{p}{[}\PY{l+m+mi}{1}\PY{o}{\PYZhy{}}\PY{n}{p\PYZus{}old}\PY{p}{,} \PY{n}{p\PYZus{}old}\PY{p}{]}\PY{p}{)}\PY{p}{;}
\end{Verbatim}


    \begin{enumerate}
\def\labelenumi{\alph{enumi}.}
\setcounter{enumi}{6}
\tightlist
\item
  Find \(p_{new}\) - \(p_{old}\) for your simulated values from part (e)
  and (f).
\end{enumerate}

    \begin{Verbatim}[commandchars=\\\{\}]
{\color{incolor}In [{\color{incolor}70}]:} \PY{n}{p\PYZus{}new\PYZus{}simulate} \PY{o}{=} \PY{n}{new\PYZus{}page\PYZus{}converted}\PY{o}{.}\PY{n}{mean}\PY{p}{(}\PY{p}{)}\PY{p}{;}
         \PY{n}{p\PYZus{}old\PYZus{}simulate} \PY{o}{=} \PY{n}{old\PYZus{}page\PYZus{}converted}\PY{o}{.}\PY{n}{mean}\PY{p}{(}\PY{p}{)}\PY{p}{;}
         
         \PY{n}{p\PYZus{}diff\PYZus{}simulate} \PY{o}{=} \PY{n}{p\PYZus{}new\PYZus{}simulate} \PY{o}{\PYZhy{}} \PY{n}{p\PYZus{}old\PYZus{}simulate}\PY{p}{;}
         
         \PY{n+nb}{print}\PY{p}{(}\PY{l+s+s1}{\PYZsq{}}\PY{l+s+s1}{p\PYZus{}new simulated: }\PY{l+s+si}{\PYZob{}\PYZcb{}}\PY{l+s+s1}{\PYZsq{}}\PY{o}{.}\PY{n}{format}\PY{p}{(}\PY{n}{p\PYZus{}new\PYZus{}simulate}\PY{p}{)}\PY{p}{)}\PY{p}{;}
         \PY{n+nb}{print}\PY{p}{(}\PY{l+s+s1}{\PYZsq{}}\PY{l+s+s1}{p\PYZus{}old simulated: }\PY{l+s+si}{\PYZob{}\PYZcb{}}\PY{l+s+s1}{\PYZsq{}}\PY{o}{.}\PY{n}{format}\PY{p}{(}\PY{n}{p\PYZus{}old\PYZus{}simulate}\PY{p}{)}\PY{p}{)}\PY{p}{;}
         \PY{n+nb}{print}\PY{p}{(}\PY{l+s+s1}{\PYZsq{}}\PY{l+s+s1}{p\PYZus{}new \PYZhy{} p\PYZus{}old: }\PY{l+s+si}{\PYZob{}\PYZcb{}}\PY{l+s+s1}{\PYZsq{}}\PY{o}{.}\PY{n}{format}\PY{p}{(}\PY{n}{p\PYZus{}diff\PYZus{}simulate}\PY{p}{)}\PY{p}{)}\PY{p}{;}
\end{Verbatim}


    \begin{Verbatim}[commandchars=\\\{\}]
p\_new simulated: 0.12047347051132062
p\_old simulated: 0.12018668171868331
p\_new - p\_old: 0.00028678879263731305

    \end{Verbatim}

    \begin{enumerate}
\def\labelenumi{\alph{enumi}.}
\setcounter{enumi}{7}
\tightlist
\item
  Simulate 10,000 \(p_{new}\) - \(p_{old}\) values using this same
  process similarly to the one you calculated in parts \textbf{a.
  through g.} above. Store all 10,000 values in \textbf{p\_diffs}.
\end{enumerate}

    \begin{Verbatim}[commandchars=\\\{\}]
{\color{incolor}In [{\color{incolor}71}]:} \PY{c+c1}{\PYZsh{}\PYZsh{} Improve perfomance by using numpy to find p\PYZus{}diffs}
         
         \PY{c+c1}{\PYZsh{} Simulate 10,000 \PYZdq{}p\PYZus{}new\PYZdq{} using np.random.binomial}
         \PY{n}{new\PYZus{}converted\PYZus{}simulation} \PY{o}{=} \PY{n}{np}\PY{o}{.}\PY{n}{random}\PY{o}{.}\PY{n}{binomial}\PY{p}{(}\PY{n}{n\PYZus{}new}\PY{p}{,} \PY{n}{p\PYZus{}new}\PY{p}{,}  \PY{l+m+mi}{10000}\PY{p}{)}\PY{o}{/}\PY{n}{n\PYZus{}new}\PY{p}{;}
         
         \PY{c+c1}{\PYZsh{} Simulate 10,000 \PYZdq{}p\PYZus{}old\PYZdq{} using np.random.binomial}
         \PY{n}{old\PYZus{}converted\PYZus{}simulation} \PY{o}{=} \PY{n}{np}\PY{o}{.}\PY{n}{random}\PY{o}{.}\PY{n}{binomial}\PY{p}{(}\PY{n}{n\PYZus{}old}\PY{p}{,} \PY{n}{p\PYZus{}old}\PY{p}{,}  \PY{l+m+mi}{10000}\PY{p}{)}\PY{o}{/}\PY{n}{n\PYZus{}old}\PY{p}{;}
         
         \PY{c+c1}{\PYZsh{} Simulate 10,000 \PYZdq{}p\PYZus{}diffs\PYZdq{}}
         \PY{n}{p\PYZus{}diffs} \PY{o}{=} \PY{n}{new\PYZus{}converted\PYZus{}simulation} \PY{o}{\PYZhy{}} \PY{n}{old\PYZus{}converted\PYZus{}simulation}\PY{p}{;}
\end{Verbatim}


    \begin{Verbatim}[commandchars=\\\{\}]
{\color{incolor}In [{\color{incolor}72}]:} \PY{c+c1}{\PYZsh{}\PYZsh{} (Slow) Original approach \PYZhy{} for loop }
         \PY{c+c1}{\PYZsh{} DO NOT USE IT. Commented out here for reference}
         
         \PY{c+c1}{\PYZsh{} p\PYZus{}diffs = [];}
         
         \PY{c+c1}{\PYZsh{} print(\PYZsq{}* Start the sampling process\PYZsq{});}
         \PY{c+c1}{\PYZsh{} for \PYZus{} in range(10000):}
             
         \PY{c+c1}{\PYZsh{}     \PYZsh{} Simulate data for new page}
         \PY{c+c1}{\PYZsh{}     new\PYZus{}page\PYZus{}converted\PYZus{}sample = np.random.choice([0, 1], n\PYZus{}new, p=[1\PYZhy{}p\PYZus{}new, p\PYZus{}new]);}
             
         \PY{c+c1}{\PYZsh{}     \PYZsh{} Simulate data for old page}
         \PY{c+c1}{\PYZsh{}     old\PYZus{}page\PYZus{}converted\PYZus{}sample = np.random.choice([0, 1], n\PYZus{}old, p=[1\PYZhy{}p\PYZus{}old, p\PYZus{}old]);}
             
         \PY{c+c1}{\PYZsh{}     \PYZsh{} Find a conversion rate for each sample}
         \PY{c+c1}{\PYZsh{}     p\PYZus{}new\PYZus{}sample = new\PYZus{}page\PYZus{}converted\PYZus{}sample.mean();}
         \PY{c+c1}{\PYZsh{}     p\PYZus{}old\PYZus{}sample = old\PYZus{}page\PYZus{}converted\PYZus{}sample.mean();}
             
         \PY{c+c1}{\PYZsh{}     \PYZsh{} Add a difference between a conversion rate of new page and old page to p\PYZus{}diffs}
         \PY{c+c1}{\PYZsh{}     p\PYZus{}diffs.append(p\PYZus{}new\PYZus{}sample \PYZhy{} p\PYZus{}old\PYZus{}sample);}
             
         \PY{c+c1}{\PYZsh{} print(\PYZsq{}* Finish the sampling process\PYZsq{});}
         \PY{c+c1}{\PYZsh{} p\PYZus{}diffs = np.array(p\PYZus{}diffs);  }
\end{Verbatim}


    \begin{enumerate}
\def\labelenumi{\roman{enumi}.}
\tightlist
\item
  Plot a histogram of the \textbf{p\_diffs}. Does this plot look like
  what you expected? Use the matching problem in the classroom to assure
  you fully understand what was computed here.
\end{enumerate}

    \textbf{Answer:}

According to the \textbf{Central Limit Theorem} , I expect the sampling
distribution of the \textbf{p\_diffs} to be normally distributed.

    \begin{Verbatim}[commandchars=\\\{\}]
{\color{incolor}In [{\color{incolor}73}]:} \PY{c+c1}{\PYZsh{} Histogram of p\PYZus{}diffs}
         \PY{n}{plt}\PY{o}{.}\PY{n}{hist}\PY{p}{(}\PY{n}{p\PYZus{}diffs}\PY{p}{)}\PY{p}{;}
\end{Verbatim}


    \begin{center}
    \adjustimage{max size={0.9\linewidth}{0.9\paperheight}}{output_59_0.png}
    \end{center}
    { \hspace*{\fill} \\}
    
    \begin{Verbatim}[commandchars=\\\{\}]
{\color{incolor}In [{\color{incolor}74}]:} \PY{c+c1}{\PYZsh{} Find the observed p\PYZus{}diff from the actual data}
         \PY{n}{df2\PYZus{}old} \PY{o}{=} \PY{n}{df2}\PY{o}{.}\PY{n}{query}\PY{p}{(}\PY{l+s+s1}{\PYZsq{}}\PY{l+s+s1}{group == }\PY{l+s+s1}{\PYZdq{}}\PY{l+s+s1}{control}\PY{l+s+s1}{\PYZdq{}}\PY{l+s+s1}{\PYZsq{}}\PY{p}{)}\PY{p}{;}
         \PY{n}{df2\PYZus{}new} \PY{o}{=} \PY{n}{df2}\PY{o}{.}\PY{n}{query}\PY{p}{(}\PY{l+s+s1}{\PYZsq{}}\PY{l+s+s1}{group == }\PY{l+s+s1}{\PYZdq{}}\PY{l+s+s1}{treatment}\PY{l+s+s1}{\PYZdq{}}\PY{l+s+s1}{\PYZsq{}}\PY{p}{)}\PY{p}{;}
         
         \PY{n}{p\PYZus{}old\PYZus{}obs} \PY{o}{=} \PY{n}{df2\PYZus{}old}\PY{p}{[}\PY{l+s+s1}{\PYZsq{}}\PY{l+s+s1}{converted}\PY{l+s+s1}{\PYZsq{}}\PY{p}{]}\PY{o}{.}\PY{n}{mean}\PY{p}{(}\PY{p}{)}\PY{p}{;}
         \PY{n}{p\PYZus{}new\PYZus{}obs} \PY{o}{=} \PY{n}{df2\PYZus{}new}\PY{p}{[}\PY{l+s+s1}{\PYZsq{}}\PY{l+s+s1}{converted}\PY{l+s+s1}{\PYZsq{}}\PY{p}{]}\PY{o}{.}\PY{n}{mean}\PY{p}{(}\PY{p}{)}\PY{p}{;}
         
         \PY{n}{p\PYZus{}diff\PYZus{}obs} \PY{o}{=} \PY{n}{p\PYZus{}new\PYZus{}obs} \PY{o}{\PYZhy{}} \PY{n}{p\PYZus{}old\PYZus{}obs}\PY{p}{;}
         
         \PY{n+nb}{print}\PY{p}{(}\PY{l+s+s1}{\PYZsq{}}\PY{l+s+s1}{Observed p\PYZus{}diff: }\PY{l+s+si}{\PYZob{}\PYZcb{}}\PY{l+s+s1}{\PYZsq{}}\PY{o}{.}\PY{n}{format}\PY{p}{(}\PY{n}{p\PYZus{}diff\PYZus{}obs}\PY{p}{)}\PY{p}{)}\PY{p}{;}
\end{Verbatim}


    \begin{Verbatim}[commandchars=\\\{\}]
Observed p\_diff: -0.0015782389853555567

    \end{Verbatim}

    \begin{Verbatim}[commandchars=\\\{\}]
{\color{incolor}In [{\color{incolor}75}]:} \PY{c+c1}{\PYZsh{} Simulate distribution under the null hypothesis}
         \PY{n}{std\PYZus{}diffs} \PY{o}{=} \PY{n}{p\PYZus{}diffs}\PY{o}{.}\PY{n}{std}\PY{p}{(}\PY{p}{)}\PY{p}{;}
         \PY{n}{null\PYZus{}vals} \PY{o}{=} \PY{n}{np}\PY{o}{.}\PY{n}{random}\PY{o}{.}\PY{n}{normal}\PY{p}{(}\PY{l+m+mi}{0}\PY{p}{,} \PY{n}{std\PYZus{}diffs}\PY{p}{,} \PY{n}{p\PYZus{}diffs}\PY{o}{.}\PY{n}{size}\PY{p}{)}
         
         \PY{c+c1}{\PYZsh{} Plot the null distribution}
         \PY{n}{plt}\PY{o}{.}\PY{n}{hist}\PY{p}{(}\PY{n}{null\PYZus{}vals}\PY{p}{)}\PY{p}{;}
         
         \PY{c+c1}{\PYZsh{} Add a vertical line to show the actual p\PYZus{}diff}
         \PY{n}{plt}\PY{o}{.}\PY{n}{axvline}\PY{p}{(}\PY{n}{p\PYZus{}diff\PYZus{}obs}\PY{p}{,} \PY{n}{c}\PY{o}{=}\PY{l+s+s1}{\PYZsq{}}\PY{l+s+s1}{red}\PY{l+s+s1}{\PYZsq{}}\PY{p}{)}\PY{p}{;}
\end{Verbatim}


    \begin{center}
    \adjustimage{max size={0.9\linewidth}{0.9\paperheight}}{output_61_0.png}
    \end{center}
    { \hspace*{\fill} \\}
    
    \begin{enumerate}
\def\labelenumi{\alph{enumi}.}
\setcounter{enumi}{9}
\tightlist
\item
  What proportion of the \textbf{p\_diffs} are greater than the actual
  difference observed in \textbf{ab\_data.csv}?
\end{enumerate}

    \begin{Verbatim}[commandchars=\\\{\}]
{\color{incolor}In [{\color{incolor}76}]:} \PY{c+c1}{\PYZsh{} The proportion of the **p\PYZus{}diffs** are greater than the actual difference }
         \PY{c+c1}{\PYZsh{} is a p\PYZhy{}value}
         
         \PY{n}{pval} \PY{o}{=} \PY{p}{(}\PY{n}{null\PYZus{}vals} \PY{o}{\PYZgt{}} \PY{n}{p\PYZus{}diff\PYZus{}obs}\PY{p}{)}\PY{o}{.}\PY{n}{mean}\PY{p}{(}\PY{p}{)}\PY{p}{;}
         \PY{n+nb}{print}\PY{p}{(}\PY{l+s+s1}{\PYZsq{}}\PY{l+s+s1}{p\PYZhy{}value: }\PY{l+s+si}{\PYZob{}\PYZcb{}}\PY{l+s+s1}{\PYZsq{}}\PY{o}{.}\PY{n}{format}\PY{p}{(}\PY{n}{pval}\PY{p}{)}\PY{p}{)}
\end{Verbatim}


    \begin{Verbatim}[commandchars=\\\{\}]
p-value: 0.9075

    \end{Verbatim}

    \begin{Verbatim}[commandchars=\\\{\}]
{\color{incolor}In [{\color{incolor}77}]:} \PY{c+c1}{\PYZsh{}\PYZsh{} The following approach can be used for a two sided hypothesis }
         \PY{c+c1}{\PYZsh{}This is not an answer the the question j. Commented out here for reference.}
         
         \PY{c+c1}{\PYZsh{} \PYZsh{} for a two sided hypothesis, we want to look at anything }
         \PY{c+c1}{\PYZsh{} \PYZsh{} more extreme from the null in both directions}
         \PY{c+c1}{\PYZsh{} null\PYZus{}mean = 0;}
         
         \PY{c+c1}{\PYZsh{} \PYZsh{} probability of a statistic lower than observed}
         \PY{c+c1}{\PYZsh{} prob\PYZus{}more\PYZus{}extreme\PYZus{}low = (null\PYZus{}vals \PYZlt{} p\PYZus{}diff\PYZus{}obs).mean()}
         
         \PY{c+c1}{\PYZsh{} \PYZsh{} probability a statistic is more extreme higher}
         \PY{c+c1}{\PYZsh{} prob\PYZus{}more\PYZus{}extreme\PYZus{}high = (null\PYZus{}mean + (null\PYZus{}mean \PYZhy{} p\PYZus{}diff\PYZus{}obs) \PYZlt{} null\PYZus{}vals).mean()}
         
         \PY{c+c1}{\PYZsh{} pval = prob\PYZus{}more\PYZus{}extreme\PYZus{}low + prob\PYZus{}more\PYZus{}extreme\PYZus{}high}
         \PY{c+c1}{\PYZsh{} print(\PYZsq{}p\PYZhy{}value: \PYZob{}\PYZcb{}\PYZsq{}.format(pval))}
\end{Verbatim}


    \begin{enumerate}
\def\labelenumi{\alph{enumi}.}
\setcounter{enumi}{10}
\tightlist
\item
  In words, explain what you just computed in part \textbf{j.}. What is
  this value called in scientific studies? What does this value mean in
  terms of whether or not there is a difference between the new and old
  pages?
\end{enumerate}

    \textbf{Answer:}

The value we computed in part \textbf{j} is a p-value.

Since the p-value (0.9) is greater than a type I error rate of 0.05,
there is no evidence that the new page is better that the old page.

    \begin{enumerate}
\def\labelenumi{\alph{enumi}.}
\setcounter{enumi}{11}
\tightlist
\item
  We could also use a built-in to achieve similar results. Though using
  the built-in might be easier to code, the above portions are a
  walkthrough of the ideas that are critical to correctly thinking about
  statistical significance. Fill in the below to calculate the number of
  conversions for each page, as well as the number of individuals who
  received each page. Let \texttt{n\_old} and \texttt{n\_new} refer the
  the number of rows associated with the old page and new pages,
  respectively.
\end{enumerate}

    \begin{Verbatim}[commandchars=\\\{\}]
{\color{incolor}In [{\color{incolor}78}]:} \PY{k+kn}{import} \PY{n+nn}{statsmodels}\PY{n+nn}{.}\PY{n+nn}{api} \PY{k}{as} \PY{n+nn}{sm}
         
         \PY{n}{convert\PYZus{}old} \PY{o}{=} \PY{n}{df2\PYZus{}old}\PY{p}{[}\PY{l+s+s1}{\PYZsq{}}\PY{l+s+s1}{converted}\PY{l+s+s1}{\PYZsq{}}\PY{p}{]}\PY{o}{.}\PY{n}{sum}\PY{p}{(}\PY{p}{)}\PY{p}{;}
         \PY{n}{convert\PYZus{}new} \PY{o}{=} \PY{n}{df2\PYZus{}new}\PY{p}{[}\PY{l+s+s1}{\PYZsq{}}\PY{l+s+s1}{converted}\PY{l+s+s1}{\PYZsq{}}\PY{p}{]}\PY{o}{.}\PY{n}{sum}\PY{p}{(}\PY{p}{)}\PY{p}{;}
         \PY{n}{n\PYZus{}old} \PY{o}{=} \PY{n}{df2\PYZus{}old}\PY{o}{.}\PY{n}{shape}\PY{p}{[}\PY{l+m+mi}{0}\PY{p}{]}\PY{p}{;}
         \PY{n}{n\PYZus{}new} \PY{o}{=} \PY{n}{df2\PYZus{}new}\PY{o}{.}\PY{n}{shape}\PY{p}{[}\PY{l+m+mi}{0}\PY{p}{]}\PY{p}{;}
\end{Verbatim}


    \begin{enumerate}
\def\labelenumi{\alph{enumi}.}
\setcounter{enumi}{12}
\tightlist
\item
  Now use \texttt{stats.proportions\_ztest} to compute your test
  statistic and p-value.
  \href{http://knowledgetack.com/python/statsmodels/proportions_ztest/}{Here}
  is a helpful link on using the built in.
\end{enumerate}

    \begin{Verbatim}[commandchars=\\\{\}]
{\color{incolor}In [{\color{incolor}79}]:} \PY{c+c1}{\PYZsh{} Since this is a one\PYZhy{}sided test, we need to specify \PYZhy{}\PYZgt{} alternative=\PYZdq{}smaller\PYZdq{}.}
         \PY{c+c1}{\PYZsh{} (the alternative hypothesis to assume the conversion rate of the new\PYZus{}page }
         \PY{c+c1}{\PYZsh{} is better than the old\PYZus{}page or p\PYZus{}new \PYZgt{} p\PYZus{}old)}
         
         \PY{n}{z\PYZus{}score}\PY{p}{,} \PY{n}{p\PYZus{}value} \PY{o}{=} \PY{n}{sm}\PY{o}{.}\PY{n}{stats}\PY{o}{.}\PY{n}{proportions\PYZus{}ztest}\PY{p}{(}\PY{p}{[}\PY{n}{convert\PYZus{}old}\PY{p}{,} \PY{n}{convert\PYZus{}new}\PY{p}{]}\PY{p}{,} \PY{p}{[}\PY{n}{n\PYZus{}old}\PY{p}{,} \PY{n}{n\PYZus{}new}\PY{p}{]}\PY{p}{,} \PY{n}{alternative}\PY{o}{=}\PY{l+s+s2}{\PYZdq{}}\PY{l+s+s2}{smaller}\PY{l+s+s2}{\PYZdq{}}\PY{p}{)}
\end{Verbatim}


    \begin{Verbatim}[commandchars=\\\{\}]
{\color{incolor}In [{\color{incolor}80}]:} \PY{n+nb}{print}\PY{p}{(}\PY{l+s+s1}{\PYZsq{}}\PY{l+s+s1}{z\PYZus{}score: }\PY{l+s+si}{\PYZob{}\PYZcb{}}\PY{l+s+s1}{\PYZsq{}}\PY{o}{.}\PY{n}{format}\PY{p}{(}\PY{n}{z\PYZus{}score}\PY{p}{)}\PY{p}{)}\PY{p}{;}
         \PY{n+nb}{print}\PY{p}{(}\PY{l+s+s1}{\PYZsq{}}\PY{l+s+s1}{p\PYZus{}value: }\PY{l+s+si}{\PYZob{}\PYZcb{}}\PY{l+s+s1}{\PYZsq{}}\PY{o}{.}\PY{n}{format}\PY{p}{(}\PY{n}{p\PYZus{}value}\PY{p}{)}\PY{p}{)}\PY{p}{;}
\end{Verbatim}


    \begin{Verbatim}[commandchars=\\\{\}]
z\_score: 1.3109241984234394
p\_value: 0.9050583127590245

    \end{Verbatim}

    \begin{enumerate}
\def\labelenumi{\alph{enumi}.}
\setcounter{enumi}{13}
\tightlist
\item
  What do the z-score and p-value you computed in the previous question
  mean for the conversion rates of the old and new pages? Do they agree
  with the findings in parts \textbf{j.} and \textbf{k.}?
\end{enumerate}

    \textbf{Answer:}

Since the z-score of 1.31 is in a range {[}-1.96, 1.96{]} which is in
the 95\% confidence and the p-value (0.90) is greater than 0.05, there
is no evidence that the new page is better that the old page.

This finding is consistent with the findings in parts \textbf{j} and
\textbf{k}.

     \#\#\# Part III - A regression approach

\texttt{1.} In this final part, you will see that the result you
acheived in the previous A/B test can also be acheived by performing
regression.

\begin{enumerate}
\def\labelenumi{\alph{enumi}.}
\tightlist
\item
  Since each row is either a conversion or no conversion, what type of
  regression should you be performing in this case?
\end{enumerate}

    \textbf{Answer:}

We can perform the lostistic regression in this case since the outcome
of the conversion will be either 0 or 1.

    \begin{enumerate}
\def\labelenumi{\alph{enumi}.}
\setcounter{enumi}{1}
\tightlist
\item
  The goal is to use \textbf{statsmodels} to fit the regression model
  you specified in part \textbf{a.} to see if there is a significant
  difference in conversion based on which page a customer receives.
  However, you first need to create a column for the intercept, and
  create a dummy variable column for which page each user received. Add
  an \textbf{intercept} column, as well as an \textbf{ab\_page} column,
  which is 1 when an individual receives the \textbf{treatment} and 0 if
  \textbf{control}.
\end{enumerate}

    \begin{Verbatim}[commandchars=\\\{\}]
{\color{incolor}In [{\color{incolor}81}]:} \PY{c+c1}{\PYZsh{} Add an intercept column}
         \PY{n}{df2}\PY{p}{[}\PY{l+s+s1}{\PYZsq{}}\PY{l+s+s1}{intercept}\PY{l+s+s1}{\PYZsq{}}\PY{p}{]} \PY{o}{=} \PY{l+m+mi}{1}\PY{p}{;}
\end{Verbatim}


    \begin{Verbatim}[commandchars=\\\{\}]
{\color{incolor}In [{\color{incolor}82}]:} \PY{c+c1}{\PYZsh{} Add as an ab\PYZus{}page column, which is 1 }
         \PY{c+c1}{\PYZsh{} when an individual receives the treatment and 0 if control.}
         
         \PY{n}{df\PYZus{}dummy} \PY{o}{=} \PY{n}{pd}\PY{o}{.}\PY{n}{get\PYZus{}dummies}\PY{p}{(}\PY{n}{df2}\PY{p}{[}\PY{l+s+s1}{\PYZsq{}}\PY{l+s+s1}{group}\PY{l+s+s1}{\PYZsq{}}\PY{p}{]}\PY{p}{)}\PY{p}{;}
         \PY{n}{df2}\PY{p}{[}\PY{l+s+s1}{\PYZsq{}}\PY{l+s+s1}{ab\PYZus{}page}\PY{l+s+s1}{\PYZsq{}}\PY{p}{]} \PY{o}{=} \PY{n}{df\PYZus{}dummy}\PY{p}{[}\PY{l+s+s1}{\PYZsq{}}\PY{l+s+s1}{treatment}\PY{l+s+s1}{\PYZsq{}}\PY{p}{]}\PY{p}{;}
         \PY{n}{df2}\PY{o}{.}\PY{n}{head}\PY{p}{(}\PY{p}{)}
\end{Verbatim}


\begin{Verbatim}[commandchars=\\\{\}]
{\color{outcolor}Out[{\color{outcolor}82}]:}    user\_id                   timestamp      group landing\_page  converted  \textbackslash{}
         0   851104  2017-01-21 22:11:48.556739    control     old\_page          0   
         1   804228  2017-01-12 08:01:45.159739    control     old\_page          0   
         2   661590  2017-01-11 16:55:06.154213  treatment     new\_page          0   
         3   853541  2017-01-08 18:28:03.143765  treatment     new\_page          0   
         4   864975  2017-01-21 01:52:26.210827    control     old\_page          1   
         
            intercept  ab\_page  
         0          1        0  
         1          1        0  
         2          1        1  
         3          1        1  
         4          1        0  
\end{Verbatim}
            
    \begin{enumerate}
\def\labelenumi{\alph{enumi}.}
\setcounter{enumi}{2}
\tightlist
\item
  Use \textbf{statsmodels} to import your regression model. Instantiate
  the model, and fit the model using the two columns you created in part
  \textbf{b.} to predict whether or not an individual converts.
\end{enumerate}

    \begin{Verbatim}[commandchars=\\\{\}]
{\color{incolor}In [{\color{incolor}83}]:} \PY{c+c1}{\PYZsh{} Only use the \PYZsq{}ab\PYZus{}page\PYZsq{} column and do not use a column for the control group}
         \PY{c+c1}{\PYZsh{} when fitting the model to make it a full matrix}
         
         \PY{n}{logit\PYZus{}mod} \PY{o}{=} \PY{n}{sm}\PY{o}{.}\PY{n}{Logit}\PY{p}{(}\PY{n}{df2}\PY{p}{[}\PY{l+s+s1}{\PYZsq{}}\PY{l+s+s1}{converted}\PY{l+s+s1}{\PYZsq{}}\PY{p}{]}\PY{p}{,} \PY{n}{df2}\PY{p}{[}\PY{p}{[}\PY{l+s+s1}{\PYZsq{}}\PY{l+s+s1}{intercept}\PY{l+s+s1}{\PYZsq{}}\PY{p}{,} \PY{l+s+s1}{\PYZsq{}}\PY{l+s+s1}{ab\PYZus{}page}\PY{l+s+s1}{\PYZsq{}}\PY{p}{]}\PY{p}{]}\PY{p}{)}\PY{p}{;}
         \PY{n}{results} \PY{o}{=} \PY{n}{logit\PYZus{}mod}\PY{o}{.}\PY{n}{fit}\PY{p}{(}\PY{p}{)}\PY{p}{;}
\end{Verbatim}


    \begin{Verbatim}[commandchars=\\\{\}]
Optimization terminated successfully.
         Current function value: 0.366118
         Iterations 6

    \end{Verbatim}

    \begin{enumerate}
\def\labelenumi{\alph{enumi}.}
\setcounter{enumi}{3}
\tightlist
\item
  Provide the summary of your model below, and use it as necessary to
  answer the following questions.
\end{enumerate}

    \begin{Verbatim}[commandchars=\\\{\}]
{\color{incolor}In [{\color{incolor}84}]:} \PY{n}{results}\PY{o}{.}\PY{n}{summary}\PY{p}{(}\PY{p}{)}
\end{Verbatim}


\begin{Verbatim}[commandchars=\\\{\}]
{\color{outcolor}Out[{\color{outcolor}84}]:} <class 'statsmodels.iolib.summary.Summary'>
         """
                                    Logit Regression Results                           
         ==============================================================================
         Dep. Variable:              converted   No. Observations:               290584
         Model:                          Logit   Df Residuals:                   290582
         Method:                           MLE   Df Model:                            1
         Date:                Tue, 10 Apr 2018   Pseudo R-squ.:               8.077e-06
         Time:                        01:00:58   Log-Likelihood:            -1.0639e+05
         converged:                       True   LL-Null:                   -1.0639e+05
                                                 LLR p-value:                    0.1899
         ==============================================================================
                          coef    std err          z      P>|z|      [0.025      0.975]
         ------------------------------------------------------------------------------
         intercept     -1.9888      0.008   -246.669      0.000      -2.005      -1.973
         ab\_page       -0.0150      0.011     -1.311      0.190      -0.037       0.007
         ==============================================================================
         """
\end{Verbatim}
            
    \begin{enumerate}
\def\labelenumi{\alph{enumi}.}
\setcounter{enumi}{4}
\tightlist
\item
  What is the p-value associated with \textbf{ab\_page}? Why does it
  differ from the value you found in the \textbf{Part II}?
  \textbf{Hint}: What are the null and alternative hypotheses associated
  with your regression model, and how do they compare to the null and
  alternative hypotheses in the \textbf{Part II}?
\end{enumerate}

    \textbf{Answer:}

The p-value associated with \textbf{ab\_page} is 0.19. This p-value is
different from the p-valu we found in \textbf{Part II} where p-value is
0.9.

The difference in p-values came from a difference in the hypothesis for
each approach where:

\textbf{Part II}: assumes a ``one-side'' hypothesis - The altenative
hypotheis is that the conversion rate of the new page is greater than
the old page (p\_new \textgreater{} p\_old). Since our concern in Part
II is which page had a higher conversion rate, a ``one-side'' hypothesis
is applied here.

\textbf{Part III}: assumes a ``two-sided'' hypothesis - The altenative
hypotheis is that the conversion rate of the new page is not equal to
the old page (p\_new != p\_old). Since we are doing a regression test in
Part III, it is concerned with if the condition had any effect at all,
so a ``two-sided'' hypothesis is applied here.

    \begin{enumerate}
\def\labelenumi{\alph{enumi}.}
\setcounter{enumi}{5}
\tightlist
\item
  Now, you are considering other things that might influence whether or
  not an individual converts. Discuss why it is a good idea to consider
  other factors to add into your regression model. Are there any
  disadvantages to adding additional terms into your regression model?
\end{enumerate}

    \textbf{Answer:}

When adding other factors into the regression model, it enables us to be
able to determine the relative influence of one or more variables to the
criterion value.

However, when adding additional terms into the regression model, we need
to ensure that they are not related to one another. Our model could be
suffered from the Multicollinearity problem when the x-variables are
correlated.

    \begin{enumerate}
\def\labelenumi{\alph{enumi}.}
\setcounter{enumi}{6}
\tightlist
\item
  Now along with testing if the conversion rate changes for different
  pages, also add an effect based on which country a user lives. You
  will need to read in the \textbf{countries.csv} dataset and merge
  together your datasets on the approporiate rows.
  \href{https://pandas.pydata.org/pandas-docs/stable/generated/pandas.DataFrame.join.html}{Here}
  are the docs for joining tables.
\end{enumerate}

Does it appear that country had an impact on conversion? Don't forget to
create dummy variables for these country columns - \textbf{Hint: You
will need two columns for the three dummy varaibles.} Provide the
statistical output as well as a written response to answer this
question.

    \begin{Verbatim}[commandchars=\\\{\}]
{\color{incolor}In [{\color{incolor}85}]:} \PY{n}{df\PYZus{}countries} \PY{o}{=} \PY{n}{pd}\PY{o}{.}\PY{n}{read\PYZus{}csv}\PY{p}{(}\PY{l+s+s1}{\PYZsq{}}\PY{l+s+s1}{countries.csv}\PY{l+s+s1}{\PYZsq{}}\PY{p}{)}\PY{p}{;}
         
         \PY{c+c1}{\PYZsh{} Join the country dataframe with df2 using the \PYZsq{}user\PYZus{}id\PYZsq{} column}
         \PY{n}{df\PYZus{}new} \PY{o}{=} \PY{n}{df2}\PY{o}{.}\PY{n}{set\PYZus{}index}\PY{p}{(}\PY{l+s+s1}{\PYZsq{}}\PY{l+s+s1}{user\PYZus{}id}\PY{l+s+s1}{\PYZsq{}}\PY{p}{)}\PY{o}{.}\PY{n}{join}\PY{p}{(}\PY{n}{df\PYZus{}countries}\PY{o}{.}\PY{n}{set\PYZus{}index}\PY{p}{(}\PY{l+s+s1}{\PYZsq{}}\PY{l+s+s1}{user\PYZus{}id}\PY{l+s+s1}{\PYZsq{}}\PY{p}{)}\PY{p}{)}\PY{p}{;}
         \PY{n}{df\PYZus{}new}\PY{o}{.}\PY{n}{head}\PY{p}{(}\PY{p}{)}
\end{Verbatim}


\begin{Verbatim}[commandchars=\\\{\}]
{\color{outcolor}Out[{\color{outcolor}85}]:}                           timestamp      group landing\_page  converted  \textbackslash{}
         user\_id                                                                  
         851104   2017-01-21 22:11:48.556739    control     old\_page          0   
         804228   2017-01-12 08:01:45.159739    control     old\_page          0   
         661590   2017-01-11 16:55:06.154213  treatment     new\_page          0   
         853541   2017-01-08 18:28:03.143765  treatment     new\_page          0   
         864975   2017-01-21 01:52:26.210827    control     old\_page          1   
         
                  intercept  ab\_page country  
         user\_id                              
         851104           1        0      US  
         804228           1        0      US  
         661590           1        1      US  
         853541           1        1      US  
         864975           1        0      US  
\end{Verbatim}
            
    \begin{Verbatim}[commandchars=\\\{\}]
{\color{incolor}In [{\color{incolor}86}]:} \PY{c+c1}{\PYZsh{} Add a new column for each country in  the country column}
         \PY{n}{df\PYZus{}new}\PY{p}{[}\PY{p}{[}\PY{l+s+s1}{\PYZsq{}}\PY{l+s+s1}{CA}\PY{l+s+s1}{\PYZsq{}}\PY{p}{,} \PY{l+s+s1}{\PYZsq{}}\PY{l+s+s1}{UK}\PY{l+s+s1}{\PYZsq{}}\PY{p}{,} \PY{l+s+s1}{\PYZsq{}}\PY{l+s+s1}{US}\PY{l+s+s1}{\PYZsq{}}\PY{p}{]}\PY{p}{]} \PY{o}{=}  \PY{n}{pd}\PY{o}{.}\PY{n}{get\PYZus{}dummies}\PY{p}{(}\PY{n}{df\PYZus{}new}\PY{p}{[}\PY{l+s+s1}{\PYZsq{}}\PY{l+s+s1}{country}\PY{l+s+s1}{\PYZsq{}}\PY{p}{]}\PY{p}{)}\PY{p}{;}
         
         \PY{n}{df\PYZus{}new}\PY{o}{.}\PY{n}{head}\PY{p}{(}\PY{p}{)}
\end{Verbatim}


\begin{Verbatim}[commandchars=\\\{\}]
{\color{outcolor}Out[{\color{outcolor}86}]:}                           timestamp      group landing\_page  converted  \textbackslash{}
         user\_id                                                                  
         851104   2017-01-21 22:11:48.556739    control     old\_page          0   
         804228   2017-01-12 08:01:45.159739    control     old\_page          0   
         661590   2017-01-11 16:55:06.154213  treatment     new\_page          0   
         853541   2017-01-08 18:28:03.143765  treatment     new\_page          0   
         864975   2017-01-21 01:52:26.210827    control     old\_page          1   
         
                  intercept  ab\_page country  CA  UK  US  
         user\_id                                          
         851104           1        0      US   0   0   1  
         804228           1        0      US   0   0   1  
         661590           1        1      US   0   0   1  
         853541           1        1      US   0   0   1  
         864975           1        0      US   0   0   1  
\end{Verbatim}
            
    \begin{Verbatim}[commandchars=\\\{\}]
{\color{incolor}In [{\color{incolor}87}]:} \PY{c+c1}{\PYZsh{} Use \PYZsq{}US\PYZsq{} as a baseline}
         
         \PY{n}{logit\PYZus{}mod2} \PY{o}{=} \PY{n}{sm}\PY{o}{.}\PY{n}{Logit}\PY{p}{(}\PY{n}{df\PYZus{}new}\PY{p}{[}\PY{l+s+s1}{\PYZsq{}}\PY{l+s+s1}{converted}\PY{l+s+s1}{\PYZsq{}}\PY{p}{]}\PY{p}{,} \PY{n}{df\PYZus{}new}\PY{p}{[}\PY{p}{[}\PY{l+s+s1}{\PYZsq{}}\PY{l+s+s1}{intercept}\PY{l+s+s1}{\PYZsq{}}\PY{p}{,} \PY{l+s+s1}{\PYZsq{}}\PY{l+s+s1}{ab\PYZus{}page}\PY{l+s+s1}{\PYZsq{}}\PY{p}{,} \PY{l+s+s1}{\PYZsq{}}\PY{l+s+s1}{CA}\PY{l+s+s1}{\PYZsq{}}\PY{p}{,} \PY{l+s+s1}{\PYZsq{}}\PY{l+s+s1}{UK}\PY{l+s+s1}{\PYZsq{}}\PY{p}{]}\PY{p}{]}\PY{p}{)}\PY{p}{;}
         \PY{n}{results2} \PY{o}{=} \PY{n}{logit\PYZus{}mod2}\PY{o}{.}\PY{n}{fit}\PY{p}{(}\PY{p}{)}\PY{p}{;}
         \PY{n}{results2}\PY{o}{.}\PY{n}{summary}\PY{p}{(}\PY{p}{)}
\end{Verbatim}


    \begin{Verbatim}[commandchars=\\\{\}]
Optimization terminated successfully.
         Current function value: 0.366113
         Iterations 6

    \end{Verbatim}

\begin{Verbatim}[commandchars=\\\{\}]
{\color{outcolor}Out[{\color{outcolor}87}]:} <class 'statsmodels.iolib.summary.Summary'>
         """
                                    Logit Regression Results                           
         ==============================================================================
         Dep. Variable:              converted   No. Observations:               290584
         Model:                          Logit   Df Residuals:                   290580
         Method:                           MLE   Df Model:                            3
         Date:                Tue, 10 Apr 2018   Pseudo R-squ.:               2.323e-05
         Time:                        01:01:06   Log-Likelihood:            -1.0639e+05
         converged:                       True   LL-Null:                   -1.0639e+05
                                                 LLR p-value:                    0.1760
         ==============================================================================
                          coef    std err          z      P>|z|      [0.025      0.975]
         ------------------------------------------------------------------------------
         intercept     -1.9893      0.009   -223.763      0.000      -2.007      -1.972
         ab\_page       -0.0149      0.011     -1.307      0.191      -0.037       0.007
         CA            -0.0408      0.027     -1.516      0.130      -0.093       0.012
         UK             0.0099      0.013      0.743      0.457      -0.016       0.036
         ==============================================================================
         """
\end{Verbatim}
            
    \begin{Verbatim}[commandchars=\\\{\}]
{\color{incolor}In [{\color{incolor}88}]:} \PY{p}{(}\PY{l+m+mi}{1}\PY{o}{/}\PY{n}{np}\PY{o}{.}\PY{n}{exp}\PY{p}{(}\PY{l+m+mf}{0.0408}\PY{p}{)}\PY{p}{,} \PY{n}{np}\PY{o}{.}\PY{n}{exp}\PY{p}{(}\PY{l+m+mf}{0.0099}\PY{p}{)}\PY{p}{)}
\end{Verbatim}


\begin{Verbatim}[commandchars=\\\{\}]
{\color{outcolor}Out[{\color{outcolor}88}]:} (0.96002111497165088, 1.0099491671175422)
\end{Verbatim}
            
    \textbf{Answer:}

Because the p-values of `CA' and `UK' are large, there is no evidence
that any of them are statistically significant.

    \begin{enumerate}
\def\labelenumi{\alph{enumi}.}
\setcounter{enumi}{7}
\tightlist
\item
  Though you have now looked at the individual factors of country and
  page on conversion, we would now like to look at an interaction
  between page and country to see if there significant effects on
  conversion. Create the necessary additional columns, and fit the new
  model.
\end{enumerate}

Provide the summary results, and your conclusions based on the results.

    \begin{Verbatim}[commandchars=\\\{\}]
{\color{incolor}In [{\color{incolor}89}]:} \PY{c+c1}{\PYZsh{} Add new columns for interactions between ab\PYZus{}page and each country}
         \PY{n}{df\PYZus{}new}\PY{p}{[}\PY{l+s+s1}{\PYZsq{}}\PY{l+s+s1}{ab\PYZus{}page\PYZus{}CA}\PY{l+s+s1}{\PYZsq{}}\PY{p}{]} \PY{o}{=} \PY{n}{df\PYZus{}new}\PY{p}{[}\PY{l+s+s1}{\PYZsq{}}\PY{l+s+s1}{ab\PYZus{}page}\PY{l+s+s1}{\PYZsq{}}\PY{p}{]}\PY{o}{*}\PY{n}{df\PYZus{}new}\PY{p}{[}\PY{l+s+s1}{\PYZsq{}}\PY{l+s+s1}{CA}\PY{l+s+s1}{\PYZsq{}}\PY{p}{]}\PY{p}{;}
         \PY{n}{df\PYZus{}new}\PY{p}{[}\PY{l+s+s1}{\PYZsq{}}\PY{l+s+s1}{ab\PYZus{}page\PYZus{}UK}\PY{l+s+s1}{\PYZsq{}}\PY{p}{]} \PY{o}{=} \PY{n}{df\PYZus{}new}\PY{p}{[}\PY{l+s+s1}{\PYZsq{}}\PY{l+s+s1}{ab\PYZus{}page}\PY{l+s+s1}{\PYZsq{}}\PY{p}{]}\PY{o}{*}\PY{n}{df\PYZus{}new}\PY{p}{[}\PY{l+s+s1}{\PYZsq{}}\PY{l+s+s1}{UK}\PY{l+s+s1}{\PYZsq{}}\PY{p}{]}\PY{p}{;}
         \PY{n}{df\PYZus{}new}\PY{p}{[}\PY{l+s+s1}{\PYZsq{}}\PY{l+s+s1}{ab\PYZus{}page\PYZus{}US}\PY{l+s+s1}{\PYZsq{}}\PY{p}{]} \PY{o}{=} \PY{n}{df\PYZus{}new}\PY{p}{[}\PY{l+s+s1}{\PYZsq{}}\PY{l+s+s1}{ab\PYZus{}page}\PY{l+s+s1}{\PYZsq{}}\PY{p}{]}\PY{o}{*}\PY{n}{df\PYZus{}new}\PY{p}{[}\PY{l+s+s1}{\PYZsq{}}\PY{l+s+s1}{US}\PY{l+s+s1}{\PYZsq{}}\PY{p}{]}\PY{p}{;}
         \PY{n}{df\PYZus{}new}\PY{o}{.}\PY{n}{head}\PY{p}{(}\PY{p}{)}
\end{Verbatim}


\begin{Verbatim}[commandchars=\\\{\}]
{\color{outcolor}Out[{\color{outcolor}89}]:}                           timestamp      group landing\_page  converted  \textbackslash{}
         user\_id                                                                  
         851104   2017-01-21 22:11:48.556739    control     old\_page          0   
         804228   2017-01-12 08:01:45.159739    control     old\_page          0   
         661590   2017-01-11 16:55:06.154213  treatment     new\_page          0   
         853541   2017-01-08 18:28:03.143765  treatment     new\_page          0   
         864975   2017-01-21 01:52:26.210827    control     old\_page          1   
         
                  intercept  ab\_page country  CA  UK  US  ab\_page\_CA  ab\_page\_UK  \textbackslash{}
         user\_id                                                                   
         851104           1        0      US   0   0   1           0           0   
         804228           1        0      US   0   0   1           0           0   
         661590           1        1      US   0   0   1           0           0   
         853541           1        1      US   0   0   1           0           0   
         864975           1        0      US   0   0   1           0           0   
         
                  ab\_page\_US  
         user\_id              
         851104            0  
         804228            0  
         661590            1  
         853541            1  
         864975            0  
\end{Verbatim}
            
    \begin{Verbatim}[commandchars=\\\{\}]
{\color{incolor}In [{\color{incolor}90}]:} \PY{c+c1}{\PYZsh{} Fit the Logistic model with interactions between ab\PYZus{}page+CA and ab\PYZus{}page+UK.}
         \PY{n}{logit\PYZus{}mod\PYZus{}interaction} \PY{o}{=} \PY{n}{sm}\PY{o}{.}\PY{n}{Logit}\PY{p}{(}\PY{n}{df\PYZus{}new}\PY{p}{[}\PY{l+s+s1}{\PYZsq{}}\PY{l+s+s1}{converted}\PY{l+s+s1}{\PYZsq{}}\PY{p}{]}\PY{p}{,} \PY{n}{df\PYZus{}new}\PY{p}{[}\PY{p}{[}\PY{l+s+s1}{\PYZsq{}}\PY{l+s+s1}{intercept}\PY{l+s+s1}{\PYZsq{}}\PY{p}{,} \PY{l+s+s1}{\PYZsq{}}\PY{l+s+s1}{ab\PYZus{}page}\PY{l+s+s1}{\PYZsq{}}\PY{p}{,} \PY{l+s+s1}{\PYZsq{}}\PY{l+s+s1}{CA}\PY{l+s+s1}{\PYZsq{}}\PY{p}{,} \PY{l+s+s1}{\PYZsq{}}\PY{l+s+s1}{UK}\PY{l+s+s1}{\PYZsq{}}\PY{p}{,}\PY{l+s+s1}{\PYZsq{}}\PY{l+s+s1}{ab\PYZus{}page\PYZus{}CA}\PY{l+s+s1}{\PYZsq{}}\PY{p}{,}  \PY{l+s+s1}{\PYZsq{}}\PY{l+s+s1}{ab\PYZus{}page\PYZus{}UK}\PY{l+s+s1}{\PYZsq{}}\PY{p}{]}\PY{p}{]}\PY{p}{)}\PY{p}{;}
         \PY{n}{results\PYZus{}interaction} \PY{o}{=} \PY{n}{logit\PYZus{}mod\PYZus{}interaction}\PY{o}{.}\PY{n}{fit}\PY{p}{(}\PY{p}{)}\PY{p}{;}
         \PY{n}{results\PYZus{}interaction}\PY{o}{.}\PY{n}{summary}\PY{p}{(}\PY{p}{)}
\end{Verbatim}


    \begin{Verbatim}[commandchars=\\\{\}]
Optimization terminated successfully.
         Current function value: 0.366109
         Iterations 6

    \end{Verbatim}

\begin{Verbatim}[commandchars=\\\{\}]
{\color{outcolor}Out[{\color{outcolor}90}]:} <class 'statsmodels.iolib.summary.Summary'>
         """
                                    Logit Regression Results                           
         ==============================================================================
         Dep. Variable:              converted   No. Observations:               290584
         Model:                          Logit   Df Residuals:                   290578
         Method:                           MLE   Df Model:                            5
         Date:                Tue, 10 Apr 2018   Pseudo R-squ.:               3.482e-05
         Time:                        01:01:11   Log-Likelihood:            -1.0639e+05
         converged:                       True   LL-Null:                   -1.0639e+05
                                                 LLR p-value:                    0.1920
         ==============================================================================
                          coef    std err          z      P>|z|      [0.025      0.975]
         ------------------------------------------------------------------------------
         intercept     -1.9865      0.010   -206.344      0.000      -2.005      -1.968
         ab\_page       -0.0206      0.014     -1.505      0.132      -0.047       0.006
         CA            -0.0175      0.038     -0.465      0.642      -0.091       0.056
         UK            -0.0057      0.019     -0.306      0.760      -0.043       0.031
         ab\_page\_CA    -0.0469      0.054     -0.872      0.383      -0.152       0.059
         ab\_page\_UK     0.0314      0.027      1.181      0.238      -0.021       0.084
         ==============================================================================
         """
\end{Verbatim}
            
    \textbf{Answer:}

Since the R square value after adding interactions to the model is
increased very slightly, the interaction between ab\_page and country
does not have significant effects on conversion.

     \#\# Conclusions

Congratulations on completing the project!

\hypertarget{gather-submission-materials}{%
\subsubsection{Gather Submission
Materials}\label{gather-submission-materials}}

Once you are satisfied with the status of your Notebook, you should save
it in a format that will make it easy for others to read. You can use
the \textbf{File -\textgreater{} Download as -\textgreater{} HTML
(.html)} menu to save your notebook as an .html file. If you are working
locally and get an error about ``No module name'', then open a terminal
and try installing the missing module using
\texttt{pip\ install\ \textless{}module\_name\textgreater{}} (don't
include the ``\textless{}'' or ``\textgreater{}'' or any words following
a period in the module name).

You will submit both your original Notebook and an HTML or PDF copy of
the Notebook for review. There is no need for you to include any data
files with your submission. If you made reference to other websites,
books, and other resources to help you in solving tasks in the project,
make sure that you document them. It is recommended that you either add
a ``Resources'' section in a Markdown cell at the end of the Notebook
report, or you can include a \texttt{readme.txt} file documenting your
sources.

\hypertarget{submit-the-project}{%
\subsubsection{Submit the Project}\label{submit-the-project}}

When you're ready, click on the ``Submit Project'' button to go to the
project submission page. You can submit your files as a .zip archive or
you can link to a GitHub repository containing your project files. If
you go with GitHub, note that your submission will be a snapshot of the
linked repository at time of submission. It is recommended that you keep
each project in a separate repository to avoid any potential confusion:
if a reviewer gets multiple folders representing multiple projects,
there might be confusion regarding what project is to be evaluated.

It can take us up to a week to grade the project, but in most cases it
is much faster. You will get an email once your submission has been
reviewed. If you are having any problems submitting your project or wish
to check on the status of your submission, please email us at
dataanalyst-project@udacity.com. In the meantime, you should feel free
to continue on with your learning journey by continuing on to the next
module in the program.


    % Add a bibliography block to the postdoc
    
    
    
    \end{document}
